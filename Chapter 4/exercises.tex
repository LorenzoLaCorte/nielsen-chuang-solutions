\documentclass[a4paper,12pt]{article}
\usepackage{amsmath}
\usepackage[margin=0.9in]{geometry}
\usepackage{braket}
\usepackage{graphicx}
\usepackage{tikz}
\usepackage{amssymb}
\usetikzlibrary{quantikz}
\begin{document}
\subsection*{Exercise 4.1}
The eigenvectors are as follows:\\
Pauli $Z$: $\ket{0}$, $\ket{1}$\\
Pauli $X$: $\ket{0}+\ket{1}$, $\ket{0}-\ket{1}$\\
Pauli $Y$: $\ket{0}+i\ket{1}$, $\ket{0}-i\ket{1}$\\
Bloch sphere representations:

\includegraphics*[scale=0.4]{4.1.png}
\subsection*{Exercise 4.2}
exp$(iAx)=\displaystyle\sum_n(iAx)^n=
\displaystyle\sum_n(-1)^nx^{2n}I+\displaystyle\sum_n(-1)^nix^nA=
\cos(x)I+i\sin{x}A$
\subsection*{Exercise 4.3}
Up to a global phase:\\
$T=\noindent\begin{bmatrix}
    e^{-i\pi/8} &0\\
    0 & e^{i\pi/8}
\end{bmatrix}=
\begin{bmatrix}
    e^{-i\frac{\pi}{4}/2} &0\\
    0 & e^{i\frac{\pi}{4}/2}
\end{bmatrix}=
R_z(\pi/4)$
\subsection*{Exercise 4.4}
First consider $R_zR_xR_z$:\\
$R_zR_xR_z=\begin{bmatrix}
    \cos{\frac{\theta}{2}}e^{-i\theta} & -i\sin{\frac{\theta}{2}}\\
    -i\sin{\frac{\theta}{2}} & \cos{\frac{\theta}{2}}e^{i\theta}
\end{bmatrix}$\\
For $\theta=\frac{\pi}{2}$:\\
$R_zR_xR_z=\frac{1}{\sqrt{2}}\begin{bmatrix}
  e^{-i\frac{\pi}{2}} &  e^{-i\frac{\pi}{2}} \\
  e^{-i\frac{\pi}{2}}  & e^{i\frac{\pi}{2}}
\end{bmatrix}$\\
Hence, by multiplying by $e^{i\frac{\pi}{2}}$ we get,\\
$e^{i\frac{\pi}{2}}R_zR_xR_z=\frac{1}{\sqrt{2}}\begin{bmatrix}
    1 &  1 \\
   1  & -1
  \end{bmatrix}=H$\\
\subsection*{Exercise 4.5}
We have $n_x^2+n_y^2+n_z^2=1$\\
$\hat{n}\cdot\vec{\sigma}=\begin{bmatrix}
    n_z& n_x-in_y\\
    n_x+in_y& n_z
\end{bmatrix}$\\
Therefore,\\
$(\hat{n}\cdot\vec{\sigma})^2=\begin{bmatrix}
    n_x^2+n_y^2+n_z^2&0\\
    0&n_x^2+n_y^2+n_z^2
\end{bmatrix}=\begin{bmatrix}
    1&0\\
    0&1
\end{bmatrix}=I$\\
Consider, $R_n(\theta)R_n(-\theta)$\\
$I=R_n(\theta)R_n(-\theta)=(\cos(\frac{\theta}{2})I-\sin(\frac{\theta}{2})\hat{n}\cdot\vec{\sigma})
(\cos(\frac{\theta}{2})I+\sin(\frac{\theta}{2})\hat{n}\cdot\vec{\sigma})=
\cos^2(\frac{\theta}{2})I+\sin^2(\frac{\theta}{2})(\hat{n}\cdot\vec{\sigma})^2=
(\cos^2(\frac{\theta}{2})+\sin^2(\frac{\theta}{2}))I=I$
\subsection*{Exercise 4.6}
First, let's show that $R_Z(x)$ rotates around the Z-axis by an angle $x$.
Consider the general state $\ket{\psi}=\begin{pmatrix}
    \cos\frac{\theta}{2}\\
    e^{i\phi}\sin\frac{\theta}{2}
\end{pmatrix}$. Then,\\
$R_Z(x)\ket{\psi}=(\cos\frac{x}{2}I-i\sin\frac{x}{2}Z)\begin{pmatrix}
    \cos\frac{\theta}{2}\\
    e^{i\phi}\sin\frac{\theta}{2}
\end{pmatrix}=
\cos\frac{x}{2}\begin{pmatrix}
    \cos\frac{\theta}{2}\\
    e^{i\phi}\sin\frac{\theta}{2}
\end{pmatrix}-i\sin\frac{x}{2}\begin{pmatrix}
    \cos\frac{\theta}{2}\\
    -e^{i\phi}\sin\frac{\theta}{2}
\end{pmatrix}=
\begin{pmatrix}
    e^{-ix/2}\cos\frac{\theta}{2}\\
    e^{ix/2}e^{i\phi}\sin\frac{\theta}{2}
\end{pmatrix}=
\begin{pmatrix}
    \cos\frac{\theta}{2}\\
    e^{i(\phi+x)}\sin\frac{\theta}{2}
\end{pmatrix}$\\
Hence, the state has been rotated by $x$ around the Z-axis. Similarly, we get
that $R_X(x)$ and $R_Y(x)$ rotate around the X and Y axis respectively.\\
We also have that,\\
$R_n(x)=\cos\frac{x}{2}I-i\sin\frac{x}{2}(n_xX+n_yY+n_ZZ)=
\cos\frac{x}{2}I-i\sin\frac{x}{2}(\sin\theta_n\cos\phi_nX+\sin\theta_n\sin\theta_nY
+\cos\theta_nZ)=R_Z(\phi_n)R_X(\theta_n)(\cos\frac{x}{2}I-i\sin\frac{x}{2}Z) R_X(\theta_n)^\dagger R_Z(\phi_n)^\dagger=
R_Z(\phi_n)R_X(\theta_n)R_Z(x) R_X(\theta_n)^\dagger R_Z(\phi_n)^\dagger$\\
Therefore, $R_n(x)$ rotates the axis of rotation to the Z axis performs the rotations
by angle $x$ and then returns the axis back to $n$, which is the same as rotating around $n$
by an angle $x$. 
\subsection*{Exercise 4.7}
$\{X,Y\}=1$ therefore, $XYX=-XXY=-Y$.\\
$XR_Y(\theta)X=X(\cos\frac{\theta}{2}I-i\sin\frac{\theta}{2}Y)X=\cos\frac{\theta}{2}I+i\sin\frac{\theta}{2}Y=R_Y(-\theta)$
\subsection*{Exercise 4.8}
Any 2x2 unitary matrix for $a^2+b^2+c^2+d^2=1$ can be written as,\\
1)$U=e^{i\alpha}\begin{bmatrix}
    a+ib&c+id\\
    -c+id& a-ib
\end{bmatrix}$\\
Consider, the given form for $U$,\\
$U=e^{i\alpha}R_n(\theta)=e^{i\alpha}\begin{bmatrix}
    \cos\frac{\theta}{2}-i\sin\frac{\theta}{2}n_z & -\sin\frac{\theta}{2}(n_y+in_x)\\
    \sin\frac{\theta}{2}(n_y-in_x)& \cos\frac{\theta}{2}+i\sin\frac{\theta}{2}n_z
\end{bmatrix}$\\
As, $n_x^2+n_y^2+n_z^2=1$ this has the same form as the general $U$, hence any 
arbitrary 2x2 unitary matrix can be written as $U=e^{i\alpha}R_n(\theta)$.\\
2) $n_z=\frac{1}{\sqrt{2}}$, $n_y=0$, $n_x=\frac{1}{\sqrt{2}}$,
$\alpha=0$ and $\theta=\pi$.\\
3) $n_x,n_y=0$, $n_z=1$, $\alpha =\theta=\frac{\pi}{4}$.
\subsection*{Exercise 4.9}
We can write,\\
$U=\begin{bmatrix}
    e^{i(\alpha-\beta/2-\delta/2)}\cos\frac{\gamma}{2} & -e^{i(\alpha-\beta/2+\delta/2)}\sin\frac{\gamma}{2}\\
    e^{i(\alpha+\beta/2-\delta/2)}\sin\frac{\gamma}{2} & e^{i(\alpha+\beta/2+\delta/2)}\cos\frac{\gamma}{2}
\end{bmatrix}=e^{i\alpha}
\begin{bmatrix}
    e^{i(-\beta/2-\delta/2)}\cos\frac{\gamma}{2} & -e^{i(-\beta/2+\delta/2)}\sin\frac{\gamma}{2}\\
    e^{i(\beta/2-\delta/2)}\sin\frac{\gamma}{2} & e^{i(\beta/2+\delta/2)}\cos\frac{\gamma}{2}
\end{bmatrix}=e^{i\alpha}
\begin{bmatrix}
    a & b\\
    -b^* & a^*
\end{bmatrix}$\\
which is the general form of a 2x2 unitary matrix as $det(U)=\cos^2\frac{\gamma}{2}+
\sin^2\frac{\gamma}{2}=1$.
\subsection*{Exercise 4.10}
$U=e^{i\alpha}R_Z(\beta)R_X(\gamma)R_Z(\delta)=\begin{bmatrix}
    e^{i(\alpha-\beta/2-\delta/2)}\cos\frac{\gamma}{2} & -ie^{i(\alpha-\beta/2+\delta/2)}\sin\frac{\gamma}{2}\\
    -ie^{i(\alpha+\beta/2-\delta/2)}\sin\frac{\gamma}{2} & e^{i(\alpha+\beta/2+\delta/2)}\cos\frac{\gamma}{2}
\end{bmatrix}$\\
which once again is a unitary matrix.
\subsection*{Exercise 4.11}
\subsection*{Exercise 4.12}
From the proof of Corollary 4.2 we have, $AXBXC=R_Z(\beta)R_Y(\gamma)R_Z(\delta)$. We can see
that $\alpha=\gamma=\delta=\frac{\pi}{2}$ and $\beta=0$ gives $H$. Hence, we can take
$A=R_Y(\frac{\pi}{4})$,$B=R_Y(-\frac{\pi}{4})R_Z(-\frac{\pi}{4})$ and $C=R_Z(\frac{\pi}{4})$.
\subsection*{Exercise 4.13}
$HXH=\frac{1}{2}\begin{bmatrix}
    1&1\\
    1&-1
\end{bmatrix}
\begin{bmatrix}
    0&1\\
    1&0
\end{bmatrix}
\begin{bmatrix}
    1&1\\
    1&-1
\end{bmatrix}=
\frac{1}{2}
\begin{bmatrix}
    2&0\\
    0&-2
\end{bmatrix}=Z$\\
$HYH=\frac{1}{2}\begin{bmatrix}
    1&1\\
    1&-1
\end{bmatrix}
\begin{bmatrix}
    0&-i\\
    i&0
\end{bmatrix}
\begin{bmatrix}
    1&1\\
    1&-1
\end{bmatrix}=
\frac{1}{2}
\begin{bmatrix}
    0&2i\\
    -2i&0
\end{bmatrix}=-Y$
$HZH=HHXHH=X$
\subsection*{Exercise 4.14}
$T=R_Z(\frac{\pi}{4})$\\
$HTH=HR_Z(\frac{\pi}{4})H=H(\cos\frac{\pi}{8}-i\sin\frac{\pi}{8}Z)H=
\cos\frac{\pi}{8}-i\sin\frac{\pi}{8}X=R_X(\frac{\pi}{4})$
\subsection*{Exercise 4.15}
(Check Errata for the sign in the second equation)\\
1)$R_{\hat{n}_2}(\beta_2)R_{\hat{n}_1}(\beta_1)=
(c_2I-is_2\hat{n}_2.\sigma)(c_1I-is_1\hat{n}_1.\sigma)=
c_1c_2I-s_1s_2(\hat{n}_2.\sigma)(\hat{n}_1.\sigma) -ic_2s_1\hat{n}_1.\sigma
-ic_1s_2\hat{n}_2.\sigma=
c_1c_2I-s_1s_2(\hat{n}_1.\hat{n}_2I+i(\hat{n}_2\times\hat{n}_1).\sigma)-ic_2s_1\hat{n}_1.\sigma
-ic_1s_2\hat{n}_2.\sigma=
(c_1c_2-s_1s_2\hat{n}_1.\hat{n}_2)I-i(c_2s_1\hat{n}_1
+c_1s_2\hat{n}_2+s_1s_2\hat{n}_2\times\hat{n}_1).\sigma$\\
Therefore,\\
$c_{12}=c_1c_2-s_1s_2\hat{n}_1.\hat{n}_2\\
s_{12}\hat{n}_{12}=c_2s_1\hat{n}_1
+c_1s_2\hat{n}_2+s_1s_2\hat{n}_2\times\hat{n}_1$\\
2)$\beta_1=\beta_2$ and $\hat{n}_1=\hat{z}$, hence $c_1=c_2=c$ and $s_1=s_2=s$.
Therefore,\\
$c_{12}=c^2-s^2\hat{z}.\hat{n}_2\\
s_{12}\hat{n}_{12}=cs(\hat{z}
+\hat{n}_2)+s^2\hat{n}_2\times\hat{z}$
\subsection*{Exercise 4.16}
$H_1=\frac{1}{\sqrt{2}}\begin{bmatrix}
    I&I\\
    I&-I
\end{bmatrix}$\\
$H_2=\frac{1}{\sqrt{2}}\begin{bmatrix}
    H&0\\
    0&H
\end{bmatrix}$
\subsection*{Exercise 4.17}
$\begin{bmatrix}
    H&0\\
    0&H
\end{bmatrix}
\begin{bmatrix}
    I&0\\
    0&Z
\end{bmatrix}
\begin{bmatrix}
    H&0\\
    0&H
\end{bmatrix}=
\begin{bmatrix}
    HH&0\\
    0&HZH
\end{bmatrix}=
\begin{bmatrix}
    I&0\\
    0&X
\end{bmatrix}=CNOT$
\subsection*{Exercise 4.18}
When the second qubit is the control we have the following representation,\\
$CZ_2=\ket{00}\bra{00}+\ket{01}\bra{01}-\ket{11}\bra{11}+\ket{10}\bra{10}=
\begin{bmatrix}
    1&0&0&0\\
    0&1&0&0\\
    0&0&1&0\\
    0&0&0&-1
\end{bmatrix}$
\subsection*{Exercise 4.19}
$\text{CNOT}\rho\text{CNOT}=\begin{bmatrix}
    I&0\\
    0&X
\end{bmatrix}
\begin{bmatrix}
    A&B\\
    C&D
\end{bmatrix}
\begin{bmatrix}
    I&0\\
    0&X
\end{bmatrix}=
\begin{bmatrix}
    A&XB\\
    XC&XDX
\end{bmatrix}$\\
$X$ only rearranges elements, hence the CNOT only rearranges the elements of $\rho$.
\subsection*{Exercise 4.20}
$(H_1\otimes H_2)\text{CNOT}(H_1\otimes H_2)=\frac{1}{2}
\begin{bmatrix}
    H&H\\
    H&-H    
\end{bmatrix}
\begin{bmatrix}
    I&0\\
    0&X    
\end{bmatrix}
\begin{bmatrix}
    H&H\\
    H&-H    
\end{bmatrix}=\frac{1}{2}
\begin{bmatrix}
    HH+HXH&HH-HXH\\
    HH-HXH&HH+HXH    
\end{bmatrix}=\frac{1}{2}
\begin{bmatrix}
    I+Z&I-Z\\
    I-Z&I+Z    
\end{bmatrix}=
\begin{bmatrix}
    1&0&0&0\\
    0&0&0&1\\
    0&0&1&0\\
    0&1&0&0
\end{bmatrix}=\text{CNOT(2\textsuperscript{nd} qubit control)}$\\
The left circuit transforms between the $\ket{0}$,$\ket{1}$ and $\ket{+}$,$\ket{-}$ basis applies a CNOT and transforms back.
Hence, the effect of the CNOT on the $\ket{\pm}\ket{\pm}$ is the same as applying the CNOT with the 2\textsuperscript{nd}
qubit as target to the state in the $\ket{0}$,$\ket{1}$ basis 
and then replacing $0$ with $+$ and $1$ with $-$. This process does indeed give the equations 4.24-4.27.
\subsection*{Exercise 4.21}
Consider all the possible inputs,\\
$\ket{00\psi}\rightarrow\ket{00\psi}\rightarrow\ket{00\psi}\rightarrow\ket{00\psi}\rightarrow\ket{00\psi}\rightarrow\ket{00\psi}\\
\ket{01\psi}\rightarrow\ket{01(V\psi)}\rightarrow\ket{01(V\psi)}\rightarrow\ket{01(V^\dagger V\psi)}\rightarrow\ket{01\psi}\rightarrow\ket{01\psi}\\
\ket{10\psi}\rightarrow\ket{10\psi}\rightarrow\ket{11\psi}\rightarrow\ket{11(V^\dagger\psi)}\rightarrow\ket{10(VV^\dagger\psi)}\rightarrow\ket{10\psi}\\
\ket{11\psi}\rightarrow\ket{11(V\psi)}\rightarrow\ket{10(V\psi)}\rightarrow\ket{11(V\psi)}\rightarrow\ket{11(VV\psi)}\rightarrow\ket{11U\psi}$\\
Hence, the circuit does perform the $C^2(U)$ operation.
\newpage
\subsection*{Exercise 4.22}
Firstly, we apply the circuit in figure 4.6 to the circuit in figure 4.8 for $V=e^{i\alpha}AXBXC$, which gives,\\
$\begin{quantikz}
    &\qw&\qw & \qw&\qw &\qw &\ctrl{1} & \qw& \qw&\qw&\qw&\qw&\ctrl{1}&\qw&\ctrl{2}&\qw&\ctrl{2}&\phase{\alpha}&\qw\\
    &\qw&\ctrl{1}&\qw&\ctrl{1}&\phase{\alpha}&\targ{}&\phase{-\alpha}&\ctrl{1}&\qw&\ctrl{1}&\qw&\targ{}&\qw&\qw&\qw&\qw&\qw&\qw\\
    &\gate{C}&\targ{}&\gate{B}&\targ{}&\gate{A}&\qw&\gate{A^\dagger}&\targ{}&\gate{B^\dagger}&\targ{}&\gate{C^\dagger}&\qw&\gate{C}&\targ{}&\gate{B}&\targ{}&\gate{A}&\qw
\end{quantikz}$\\
We move the 6\textsuperscript{th} CNOT left from the 4\textsuperscript{th} one, which involves adding CNOTs
after the 4\textsuperscript{th} and 5\textsuperscript{th} CNOTs from first to third qubit, which is due to,\\
$\begin{quantikz}
    &\qw&\ctrl{1}&\qw\\
    &\ctrl{1}&\targ{}&\qw\\
    &\targ{}&\qw&\qw
\end{quantikz}=
\begin{quantikz}
    &\ctrl{1}&\qw&\ctrl{2}&\qw\\
    &\targ{}&\ctrl{1}&\qw&\qw\\
    &\qw&\targ{}&\targ{}&\qw
\end{quantikz}$\\

and can be checked by considering all the possible inputs. We also see that $AA^\dagger=CC^\dagger=I$.
Hence we get the following circuit.\\
$\begin{quantikz}
    &\qw&\qw & \qw&\qw &\phase{\alpha}\gategroup[wires=2, steps=4, style={dashed}]{} &\ctrl{1} & \qw& \ctrl{1}&\qw&\ctrl{2}&\qw&\qw&\ctrl{2}&\ctrl{2}&\qw&\ctrl{2}&\qw&\qw\\
    &\qw&\ctrl{1}&\qw&\ctrl{1}&\phase{\alpha}&\targ{}&\phase{-\alpha}&\targ{}&\ctrl{1}&\qw&\qw&\ctrl{1}&\qw&\qw&\qw&\qw&\qw&\qw\\
    &\gate{C}&\targ{}&\gate{B}&\targ{}&\qw&\qw&\qw&\qw&\targ{}&\targ{}&\gate{B^\dagger}&\targ{}&\targ&\qw&\targ{}&\gate{B}&\targ{}&\gate{A}&\qw
\end{quantikz}$\\

The dashed section is diagonal hence commutes with the rest of the components therefore can be moved to the end
for convenience.\\
$\begin{quantikz}
    &\qw&\qw & \qw&\qw &\qw&\ctrl{2}&\qw&\qw&\ctrl{2}&\ctrl{2}&\qw&\ctrl{2}&\qw&\phase{\alpha} &\ctrl{1} & \qw& \ctrl{1}&\qw\\
    &\qw&\ctrl{1}&\qw&\ctrl{1}&\ctrl{1}&\qw&\qw&\ctrl{1}&\qw&\qw&\qw&\qw&\qw&\phase{\alpha}&\targ{}&\phase{-\alpha}&\targ{}&\qw\\
    &\gate{C}&\targ{}&\gate{B}&\targ{}&\targ{}&\targ{}&\gate{B^\dagger}&\targ{}&\targ&\qw&\targ{}&\gate{B}&\targ{}&\gate{A}&\qw&\qw&\qw&\qw&\qw
\end{quantikz}$\\

CNOTs 2 and 3, and 6 and 7 cancel each other, hence we're left with the circuit.\\
$\begin{quantikz}
    &\qw&\qw & \qw&\ctrl{2}&\qw&\qw&\qw&\ctrl{2}&\qw&\phase{\alpha} &\ctrl{1} & \qw& \ctrl{1}&\qw\\
    &\qw&\ctrl{1}&\qw&\qw&\qw&\ctrl{1}&\qw&\qw&\qw&\phase{\alpha}&\targ{}&\phase{-\alpha}&\targ{}&\qw\\
    &\gate{C}&\targ{}&\gate{B}&\targ{}&\gate{B^\dagger}&\targ{}&\gate{B}&\targ{}&\gate{A}&\qw&\qw&\qw&\qw&\qw
\end{quantikz}$\\

This has 8 single qubit gates and 6 CNOTs as desired.\\
The circuit performs the operation, $AXBXB^\dagger XBXC=(VC^\dagger)B^\dagger(A^\dagger V)=V(ABC)^\dagger =V^2=U$.
Therefore, the circuit performs the $C^2(U)$ operation.
\subsection*{Exercise 4.23}
For $U=R_X(\theta)$ from corollary 4.2 we can see that $A=H$, $B=R_Z(-\frac{\theta}{2})$, $C=R_Z(\frac{\theta}{2})H$, which gives
$ABC=I$ and $AXBXC=HXR_Z(-\frac{\theta}{2})XR_Z(\frac{\theta}{2})H=HXXR_Z(\frac{\theta}{2})R_Z(\frac{\theta}{2})H=R_X(\theta)$.\\
For $U=R_Y(\theta)$ we can take $A=I$, $B=R_Y(-\frac{\theta}{2})$ and $C=R_Y(\frac{\theta}{2})$, which gives
$ABC=I$ and $AXBXC=XR_Y(-\frac{\theta}{2})XR_Y(\frac{\theta}{2})=XXR_Y(\frac{\theta}{2})R_Y(\frac{\theta}{2})=R_Y(\theta)$.
\subsection*{Exercise 4.24}
To verify the circuit we consider the state after each operation. Let the initial state be $\ket{xyz}$, where 
$x,y,z\in\{0,1\}$.\\
\begingroup
\allowdisplaybreaks
\begin{align*}
    \ket{xyz}\\
\big\downarrow &H_3\\
\frac{1}{\sqrt{2}}(\ket{xy0}+&(-1)^z\ket{xy1})\\
\big\downarrow &\text{CNOT}_{23}\\
\frac{1}{\sqrt{2}}(\ket{xyy} +&(-1)^z\ket{xy\bar{y}})\\
\big\downarrow &T^\dagger_3 \\
\frac{1}{\sqrt{2}}(e^{-iy\pi/4}\ket{xyy} +&e^{-i\bar{y}\pi/4}(-1)^z\ket{xy\bar{y}})\\
\big\downarrow &\text{CNOT}_{13}\\
\frac{1}{\sqrt{2}}(e^{-iy\pi/4}\ket{xy(y\oplus x)} +&e^{-i\bar{y}\pi/4}(-1)^z\ket{xy(\bar{y}\oplus x)})\\
\big\downarrow & T_3\\
\frac{1}{\sqrt{2}}(e^{i(y\oplus x - y)\pi/4}\ket{xy(y\oplus x)} +&e^{i(\bar{y}\oplus x -\bar{y})\pi/4}(-1)^z\ket{xy(\bar{y}\oplus x)})\\
\big\downarrow &\text{CNOT}_{23}\\
\frac{1}{\sqrt{2}}(e^{i(y\oplus x - y)\pi/4}\ket{xyx} +&e^{i(\bar{y}\oplus x -\bar{y})\pi/4}(-1)^z\ket{xy\bar{x}})\\
\big\downarrow & T^\dagger_3 \\
\frac{1}{\sqrt{2}}(e^{i(y\oplus x - y-x)\pi/4}\ket{xyx} +&e^{i(\bar{y}\oplus x -\bar{y}-\bar{x})\pi/4}(-1)^z\ket{xy\bar{x}})\\
\big\downarrow &\text{CNOT}_{13}\\
\frac{1}{\sqrt{2}}(e^{i(y\oplus x - y-x)\pi/4}\ket{xy0} +&e^{i(\bar{y}\oplus x -\bar{y}-\bar{x})\pi/4}(-1)^z\ket{xy1})\\
\big\downarrow & T_2^\dagger T_3\\
\frac{1}{\sqrt{2}}(e^{i(y\oplus x - 2y-x)\pi/4}\ket{xy0} +&e^{i(\bar{y}\oplus x -\bar{x})\pi/4}(-1)^z\ket{xy1})\\
\big\downarrow &\text{CNOT}_{12}H_3\\
\frac{1}{2}(e^{i(y\oplus x - 2y-x)\pi/4}+e^{i(\bar{y}\oplus x -\bar{x})\pi/4})\ket{x(y\oplus x)0} 
+&\frac{1}{2}(e^{i(y\oplus x - 2y-x)\pi/4}-e^{i(\bar{y}\oplus x -\bar{x})\pi/4}(-1)^z)\ket{x(y\oplus x)1}\\
\big\downarrow & T_2^\dagger\\
\frac{1}{2}(e^{i(-2y-x)\pi/4}+e^{i((1-2y)(1-2x) -\bar{x})\pi/4})\ket{x(y\oplus x)0} 
+&\frac{1}{2}(e^{i(-2y-x)\pi/4}-e^{i((1-2y)(1-2x) -\bar{x})\pi/4}(-1)^z)\ket{x(y\oplus x)1}\\
\big\downarrow &\text{CNOT}_{23}\\
\frac{1}{2}(e^{i(-2y-x)\pi/4}+e^{i((1-2y)(1-2x) -\bar{x})\pi/4})\ket{xy0} 
+&\frac{1}{2}(e^{i(-2y-x)\pi/4}-e^{i((1-2y)(1-2x) -\bar{x})\pi/4}(-1)^z)\ket{xy1}\\
\big\downarrow &T_1S_2\\
\frac{1}{2}(1+e^{i((1-2y)(1-2x) -1+2x+2y)\pi/4})\ket{xy0} 
+&\frac{1}{2}(1-e^{i((1-2y)(1-2x) -1+2x+2y)\pi/4}(-1)^z)\ket{xy1}\\
&=\\
\frac{1}{2}(1+(-1)^{xy+z})\ket{xy0} 
+&\frac{1}{2}(1-(-1)^{(xy+z)})\ket{xy1}\\
\end{align*}
\endgroup
If $x,y=1$ then we get $\ket{xy\bar{z}}$ and $\ket{xyz}$ otherwise, hence the circuit does indeed implement the
Toffoli gate.
\subsection*{Exercise 4.25}
1)\\
$\begin{quantikz}
    &\ctrl{1}&\ctrl{1}&\ctrl{1}&\qw\\
    &\ctrl{1}&\targ{}&\ctrl{1}&\qw\\
    &\targ{}&\ctrl{-1}&\targ{}&\qw
\end{quantikz}$\\
2)\\
$\begin{quantikz}
    &\qw&\ctrl{1}&\qw&\qw\\
    &\ctrl{1}&\targ{}&\ctrl{1}&\qw\\
    &\targ{}&\ctrl{-1}&\targ{}&\qw
\end{quantikz}$\\
If the first qubit is $0$, then the Toffoli just performs the identity, hence the CNOTs cancel leading to an overall
identity. If the first qubit is $1$, then the Toffoli performs a CNOT on the last 2 qubits, which overall performs the
SWAP operation.\\
3) For the Toffoli we take $V=\frac{(1-i)(1+iX)}{2}$, which gives $V^2=X$. Then, after changing the order of the 
last 2 qubits the circuit is,\\
$\begin{quantikz}
    &\qw&\qw&\ctrl{1}&\qw&\ctrl{1}&\ctrl{2}&\qw&\qw\\
    &\targ{}&\ctrl{1}&\targ{}&\ctrl{1}&\targ{}&\qw&\targ{}&\qw\\
    &\ctrl{-1}&\gate{V}&\qw&\gate{V^\dagger}&\qw&\gate{V}&\ctrl{-1}&\qw
\end{quantikz}$\\
Which simplifies to,\\
$\begin{quantikz}
    &\qw&\ctrl{1}&\qw&\ctrl{1}&\ctrl{2}&\qw&\qw\\
    &\gate[2]{CV_{23}C_{32}}&\targ{}&\ctrl{1}&\targ{}&\qw&\targ{}&\qw\\
    &&\qw&\gate{V^\dagger}&\qw&\gate{V}&\ctrl{-1}&\qw
\end{quantikz}$\\
4)As $V$ is unitary the $CV_{13}$ gate commutes with the $CV^\dagger_{23}$ and $C_{12}$ gates, hence we can move it to the left
of $CV^\dagger_{23}$. Afterwards, the last to CNOTs commute as well, hence we get\\
$\begin{quantikz}
    &\qw&\ctrl{1}&\ctrl{2}&\qw&\qw&\ctrl{1}&\qw\\
    &\gate[2]{CV_{23}C_{32}}&\targ{}&\qw&\ctrl{1}&\targ{}&\targ{}&\qw\\
    &&\qw&\gate{V}&\gate{V^\dagger}&\ctrl{-1}&\qw&\qw
\end{quantikz}=
\begin{quantikz}
    &\qw&\ctrl{1}&\ctrl{2}&\qw&\ctrl{1}&\qw\\
    &\gate[2]{CV_{23}C_{32}}&\targ{}&\qw&\gate[2]{C_{32}CV^\dagger_{23}}&\targ{}&\qw\\
    &&\qw&\gate{V}&&\qw&\qw
\end{quantikz}$\\
which contains 5 two-qubit gates.
\subsection*{Exercise 4.26}
Consider all the possible controls,\\
$\ket{00t}\rightarrow\ket{00(R_Y(\pi/4)R_Y(\pi/4)R_Y(-\pi/4)R_Y(-\pi/4))t}=\ket{00t}$\\
$\ket{01t}\rightarrow\ket{01(R_Y(\pi/4)XR_Y(\pi/4)R_Y(-\pi/4)XR_Y(-\pi/4))t}=\\
\ket{01(R_Y(\pi/4)R_Y(-\pi/4)XXR_Y(\pi/4)R_Y(-\pi/4))t}=\ket{01t}$\\
$\ket{10t}\rightarrow\ket{10(R_Y(\pi/4)R_Y(\pi/4)XR_Y(-\pi/4)R_Y(-\pi/4))t}=\\
\ket{10(R_Y(\pi/4)R_Y(\pi/4)R_Y(\pi/4)R_Y(\pi/4)X)t}=
\ket{10(R_Y(\pi)X)t}=-\ket{10(Zt)}$\\
$\ket{11t}\rightarrow\ket{11(R_Y(\pi/4)XR_Y(\pi/4)XR_Y(-\pi/4)XR_Y(-\pi/4))t}=\\
\ket{11(R_Y(\pi/4)R_Y(-\pi/4)R_Y(-\pi/4)R_Y(\pi/4)X)t}=\ket{11(Xt)}$\\
Hence, the circuit does indeed implement the Toffoli gate, with the angle for the phase factor being,\\
$\theta(c_1,c_2,t)=\begin{cases}
    \pi,&\text{for } (c_1,c_2,t)=(1,0,0)\\
    0, &\text{otherwise}
\end{cases}$
\subsection*{Exercise 4.27}
We can write the matrix as,\\
$U=\ket{000}\bra{000}+\ket{010}\bra{001}+\ket{011}\bra{010}+\ket{100}\bra{011}+\ket{101}\bra{100}
+\ket{110}\bra{101}+\ket{111}\bra{110}+\ket{001}\bra{111}$\\
Which implies the following inputs and outputs\\
$\ket{000}\rightarrow\ket{000}\\
\ket{001}\rightarrow\ket{010}\\
\ket{101}\rightarrow\ket{110}\\
\ket{010}\rightarrow\ket{011}\\
\ket{011}\rightarrow\ket{100}\\
\ket{100}\rightarrow\ket{101}\\
\ket{110}\rightarrow\ket{111}\\
\ket{111}\rightarrow\ket{001}$\\
Firstly, we can notice that the first qubit only changes when both the other ones are set,
hence we start the circuit with a Toffoli(2,3,1) with the first qubit as target. Looking at
$\ket{001}\rightarrow\ket{010}$, $\ket{101}\rightarrow\ket{110}$, $\ket{010}\rightarrow\ket{011}$,
$\ket{110}\rightarrow\ket{111}$ and $\ket{111}\rightarrow\ket{001}$ we can see that $C_{23}C_{32}$
after the Toffoli gives the desired results. For $\ket{011}\rightarrow\ket{100}$ and 
$\ket{100}\rightarrow\ket{101}$ we require a $C_{13}$ after the rest of the gates. However, 
this changes $\ket{110}\rightarrow\ket{111}$ to $\ket{110}\rightarrow\ket{110}$, hence we apply a 
Toffoli(1,2,3), as no other final state has both first and second qubit set. Hence the circuit will be,\\
$\begin{quantikz}
    &\targ{}&\qw&\qw&\ctrl{2}&\ctrl{1}&\qw\\
    &\ctrl{-1}&\targ{}&\ctrl{1}&\qw&\ctrl{1}&\qw\\
    &\ctrl{-1}&\ctrl{-1}&\targ{}&\targ{}&\targ{}&\qw
\end{quantikz}$
\subsection*{Exercise 4.28}
Analogous to the circuit for $C^2(U)$ we have,\\
$\begin{quantikz}
    &\qw&\ctrl{1}&\qw&\ctrl{1}&\ctrl{1}&\qw\\
    &\qw&\ctrl{1}&\qw&\ctrl{1}&\ctrl{1}&\qw\\
    &\qw&\ctrl{1}&\qw&\ctrl{1}&\ctrl{1}&\qw\\
    &\qw&\ctrl{1}&\qw&\ctrl{1}&\ctrl{2}&\qw\\
    &\ctrl{1}&\targ{}&\ctrl{1}&\targ{}&\qw&\qw\\
    &\gate{V}&\qw&\gate{V^\dagger}&\qw&\gate{V}&\qw
\end{quantikz}$
\subsection*{Exercise 4.29}
A recursive circuit analogous to figure 4.6, using $A=R_Z(\frac{\pi}{2})$,
$B=R_Z(-\frac{\pi}{2})R_Y(-\frac{\pi}{2})$ and $C=R_Y(\frac{\pi}{2})$. The circuit is,\\
$\begin{quantikz}
    &\qwbundle{n-1}&\ctrl{1}&\qw\\
    &\qw&\ctrl{1}&\qw\\
    &\qw&\targ{}&\qw
\end{quantikz}=
\begin{quantikz}
   &\qwbundle{n-1}&\ctrl{1}&\qw&\ctrl{1}&\qw&\qw\\
   &\ctrl{1}&\targ{}&\ctrl{1}&\targ{}&\ctrl{1}&\qw\\
   &\gate{C}&\qw&\gate{B}&\qw&\gate{A}&\qw
\end{quantikz}$\\
Each recursion requires $O(n)$ gates, hence the overall cost is
$\sum_nO(n)=O(n^2)$.\\
For more details,  arXiv:quant-ph/9503016v1.
\subsection*{Exercise 4.30}
A recursive circuit can be used for this with $V^2=U$, which analogous with figure 4.8 is,\\
$\begin{quantikz}
    &\qwbundle{n-1}&\ctrl{1}&\qw\\
    &\qw&\ctrl{1}&\qw\\
    &\qw&\gate{U}&\qw
\end{quantikz}=
\begin{quantikz}
   &\qwbundle{n-1}&\ctrl{1}&\qw&\ctrl{1}&\ctrl{2}&\qw\\
   &\ctrl{1}&\targ{}&\ctrl{1}&\targ{}&\qw&\qw\\
   &\gate{V}&\qw&\gate{V^\dagger}&\qw&\gate{V}&\qw
\end{quantikz}$\\
The cost of the CV gates is $O(1)$. Let the cost of the $C^n(U)$ be $C_n$. The cost of the
$C^{n-1}(X)$ is $O(n)$, hence the total cost from recursion will be $C_n=C_{n-1}+O(n)=O(n^2)$.\\
For more details,  arXiv:quant-ph/9503016v1.
\subsection*{Exercise 4.31}
$CX_1C=
\begin{bmatrix}
    I&0\\
    0&X    
\end{bmatrix}
\begin{bmatrix}
    0&I\\
    I&0    
\end{bmatrix}
\begin{bmatrix}
    I&0\\
    0&X    
\end{bmatrix}=
\begin{bmatrix}
    I&0\\
    0&X    
\end{bmatrix}
\begin{bmatrix}
    0&X\\
    I&0    
\end{bmatrix}=
\begin{bmatrix}
    0&X\\
    X&0    
\end{bmatrix}=X_1X_2
$\\
$CY_1C=
\begin{bmatrix}
    I&0\\
    0&X    
\end{bmatrix}
\begin{bmatrix}
    0&-iI\\
    iI&0    
\end{bmatrix}
\begin{bmatrix}
    I&0\\
    0&X    
\end{bmatrix}=
\begin{bmatrix}
    I&0\\
    0&X    
\end{bmatrix}
\begin{bmatrix}
    0&-iX\\
    iI&0    
\end{bmatrix}=
\begin{bmatrix}
    0&-iX\\
    iX&0    
\end{bmatrix}=Y_1X_2
$\\
$CZ_1C=
\begin{bmatrix}
    I&0\\
    0&X    
\end{bmatrix}
\begin{bmatrix}
    I&0\\
    0&-I    
\end{bmatrix}
\begin{bmatrix}
    I&0\\
    0&X    
\end{bmatrix}=
\begin{bmatrix}
    I&0\\
    0&X    
\end{bmatrix}
\begin{bmatrix}
    I&0\\
    0&-X    
\end{bmatrix}=
\begin{bmatrix}
    I&0\\
    0&-I    
\end{bmatrix}=Z_1
$\\
$CX_2C=
\begin{bmatrix}
    I&0\\
    0&X    
\end{bmatrix}
\begin{bmatrix}
    X&0\\
    0&X    
\end{bmatrix}
\begin{bmatrix}
    I&0\\
    0&X    
\end{bmatrix}=
\begin{bmatrix}
    I&0\\
    0&X    
\end{bmatrix}
\begin{bmatrix}
    X&0\\
    0&I    
\end{bmatrix}=
\begin{bmatrix}
    X&0\\
    0&X    
\end{bmatrix}=X_2
$\\$CY_2C=
\begin{bmatrix}
    I&0\\
    0&X    
\end{bmatrix}
\begin{bmatrix}
    Y&0\\
    0&Y    
\end{bmatrix}
\begin{bmatrix}
    I&0\\
    0&X    
\end{bmatrix}=
\begin{bmatrix}
    I&0\\
    0&X    
\end{bmatrix}
\begin{bmatrix}
    Y&0\\
    0&YX    
\end{bmatrix}=
\begin{bmatrix}
    Y&0\\
    0&-Y    
\end{bmatrix}=Z_1Y_2
$\\
$CZ_2C=
\begin{bmatrix}
    I&0\\
    0&X    
\end{bmatrix}
\begin{bmatrix}
    Z&0\\
    0&Z    
\end{bmatrix}
\begin{bmatrix}
    I&0\\
    0&X    
\end{bmatrix}=
\begin{bmatrix}
    I&0\\
    0&X    
\end{bmatrix}
\begin{bmatrix}
    Z&0\\
    0&ZX    
\end{bmatrix}=
\begin{bmatrix}
    Z&0\\
    0&-Z    
\end{bmatrix}=Z_1Z_2
$\\
$R_{Z,1}(\theta)C=(\cos{\frac{\theta}{2}}I-i\sin{\frac{\theta}{2}}Z_1)C=
\cos{\frac{\theta}{2}}CI-i\sin{\frac{\theta}{2}}CZ_1=CR_{Z,1}(\theta)$\\
$R_{X,2}(\theta)C=(\cos{\frac{\theta}{2}}I-i\sin{\frac{\theta}{2}}X_2)C=
\cos{\frac{\theta}{2}}CI-i\sin{\frac{\theta}{2}}CZ_2=CR_{X,2}(\theta)$
\subsection*{Exercise 4.32}
The state after measurement is
$\frac{P_0\rho P_0}{tr(P_0\rho P_0)}$ with probability
$tr(P_0\rho P_0)$ for measurement result $0$ and 
$\frac{P_1\rho P_1}{tr(P_1\rho P_1)}$ with probability
$tr(P_1\rho P_1)$ for measurement result $1$. Hence,\\
$\rho^\prime=tr(P_0\rho P_0)\frac{P_0\rho P_0}{tr(P_0\rho P_0)}
+tr(P_1\rho P_1)\frac{P_1\rho P_1}{tr(P_1\rho P_1)}=
P_0\rho P_0 +P_1\rho P_1$\\
$tr_2(\rho)=\displaystyle \sum_i \bra{i_2}\rho\ket{i_2}$\\
$tr_2(\rho^\prime)=\displaystyle \sum_i \bra{i_2}\rho^\prime\ket{i_2}=
\displaystyle \sum_{i,j} \bra{i_2}(\ket{j_2}\bra{j_2}\rho\ket{j_2}\bra{j_2})\ket{i_2}=
\sum_{i,j} \delta_{ij}\bra{j_2}\rho\ket{j_2}=\sum_i \bra{i_2}\rho\ket{i_2}$\\
Therefore, $tr_2(\rho)=tr_2(\rho^\prime)$.
\subsection*{Exercise 4.33}
The circuit performs the following,\\
$\ket{xy}\xrightarrow{\text{CNOT}}\ket{x(y\oplus x)}\xrightarrow{H_1}
\frac{1}{\sqrt{2}}(\ket{0(y\oplus x)}+(-1)^x\ket{1(y\oplus x)})$\\
Hence, the possible input outputs are,\\
$\ket{00}\rightarrow\frac{1}{\sqrt{2}}(\ket{00}+\ket{10})$\\
$\ket{01}\rightarrow\frac{1}{\sqrt{2}}(\ket{01}+\ket{11})$\\
$\ket{10}\rightarrow\frac{1}{\sqrt{2}}(\ket{01}-\ket{11})$\\
$\ket{11}\rightarrow\frac{1}{\sqrt{2}}(\ket{00}-\ket{10})$\\
Hence, the circuits effect on the Bell states is as follows,\\
$\frac{1}{\sqrt{2}}(\ket{00}+\ket{11})\rightarrow \ket{00}$\\
$\frac{1}{\sqrt{2}}(\ket{00}-\ket{11})\rightarrow \ket{10}$\\
$\frac{1}{\sqrt{2}}(\ket{01}+\ket{10})\rightarrow \ket{01}$\\
$\frac{1}{\sqrt{2}}(\ket{01}-\ket{10})\rightarrow \ket{11}$\\
Therefore, by performing a measurement in the computational basis
at the end of the circuit, we are overall performing a bell measurement.\\
The state $\ket{\psi}$ after the measurement is given by 
$\frac{M^\dagger_m\ket{\psi}}{\sqrt{\bra{\psi}M^\dagger_mM_m\ket{\psi}}}$. Hence,
comparing with the effect of the circuit on the bell states we see that
the measurement elements are,\\
$M_0=(\ket{00}+\ket{11})(\bra{00}+\bra{11})$\\
$M_1=(\ket{00}-\ket{11})(\bra{00}-\bra{11})$\\
$M_2=(\ket{01}+\ket{10})(\bra{01}+\bra{10})$\\
$M_3=(\ket{01}-\ket{10})(\bra{01}-\bra{10})$\\
As this are projective measurements the POVM elements are equal
to these.
\subsection*{Exercise 4.34}
This can be done by using a controlled gate to entangle the system to 
a qubit whose measurement will collapse the state into the $+1$ or $-1$
eigenbasis, while will also give us the state on the original qubit.
The circuit performs the following,\\
$\ket{0\psi_{in}}\xrightarrow{H_1}\frac{1}{\sqrt{2}}
(\ket{0\psi_{in}}+\ket{1\psi_{in}})\xrightarrow{CU}
\frac{1}{\sqrt{2}}(\ket{0\psi_{in}}+\ket{1(U\psi_{in})})\xrightarrow{H_1}
\frac{1}{2}(\ket{0\psi_{in}}+\ket{1\psi_{in}}+\ket{0(U\psi_{in})}-\ket{1(U\psi_{in})})=
\frac{1}{2}(\ket{0}(I+U)\ket{\psi_{in}}+\ket{1}(I-U)\ket{\psi_{in}})$\\
If the measurement value is $0$ then the state is 
$\ket{\psi_{out}}=(I+U)\ket{\psi_{in}}$ for which
$U\ket{\psi_{out}}=U(I+U)\ket{\psi_{in}}=(I+U)\ket{\psi_{in}}$,
therefore the result with eigenvalue $+1$ has taken place.\\
If the measurement value is $1$ then the state is
$\ket{\psi_{out}}=(I-U)\ket{\psi_{in}}$ for which
$U\ket{\psi_{out}}=U(I-U)\ket{\psi_{in}}=-(I-U)\ket{\psi_{in}}$,
therefore the result with eigenvalue $-1$ has taken place.
\subsection*{Exercise 4.35}
Let the system be in the state $a\ket{0\psi}+b\ket{1\psi}$. Then
the effect of the circuits are,\\
1)$a\ket{0\psi}+b\ket{1\psi}\xrightarrow{CU} 
a\ket{0\psi}+b\ket{1(U\psi)}\xrightarrow{\text{Mes.}}
0$ with $p=|a|^2$ and state $\ket{\psi}$ or $1$ with
$p=|b|^2$ and state $U\ket{\psi}$\\
2)$a\ket{0\psi}+b\ket{1\psi}\xrightarrow{\text{Mes.}}
0$ with $p=|a|^2$ or $1$ with
$p=|b|^2$, both with state $\ket{\psi}\xrightarrow{CU}
\ket{\psi}$ with $p=|a|^2$ or $U\ket{\psi}$ with
$p=|b|^2$\\
Hence, the two circuits perform the same operation.
\subsection*{Exercise 4.36}
Consider all the possible input states.\\
If $x=00$ then nothing needs to be applied.\\
If $x=01$ we have for $y$,\\
$\ket{00}\rightarrow\ket{01}$,
$\ket{01}\rightarrow\ket{10}$,
$\ket{10}\rightarrow\ket{11}$,
and $\ket{11}\rightarrow\ket{00}$. The circuit for this is,\\
$\begin{quantikz}
    \lstick{$\ket{x_1}$}&\octrl{1}&\octrl{1}&\qw\\
    \lstick{$\ket{x_2}$}&\ctrl{2}&\ctrl{1}&\qw\\
    \lstick{$\ket{y_1}$}&\qw&\targ{}&\qw\\
    \lstick{$\ket{y_2}$}&\targ{}&\octrl{-1}&\qw 
\end{quantikz}$\\
If $x=10$ we have for $y$,\\
$\ket{00}\rightarrow\ket{10}$,
$\ket{01}\rightarrow\ket{11}$,
$\ket{10}\rightarrow\ket{00}$,
and $\ket{11}\rightarrow\ket{01}$. The circuit for this is,\\
$\begin{quantikz}
    \lstick{$\ket{x_1}$}&\ctrl{1}&\qw\\
    \lstick{$\ket{x_2}$}&\octrl{1}&\qw\\
    \lstick{$\ket{y_1}$}&\targ{}&\qw\\
    \lstick{$\ket{y_2}$}&\qw&\qw
\end{quantikz}$\\
If $x=01$ we have for $y$,\\
$\ket{00}\rightarrow\ket{11}$,
$\ket{01}\rightarrow\ket{00}$,
$\ket{10}\rightarrow\ket{01}$,
and $\ket{11}\rightarrow\ket{10}$. The circuit for this is,\\
$\begin{quantikz}
   \lstick{$\ket{x_1}$} &\ctrl{1}&\ctrl{1}&\qw\\
   \lstick{$\ket{x_2}$} &\ctrl{2}&\ctrl{1}&\qw\\
   \lstick{$\ket{y_1}$} &\qw&\targ{}&\qw\\
   \lstick{$\ket{y_2}$} &\targ{}&\ctrl{-2}&\qw 
\end{quantikz}$\\
Hence, the full circuit is,\\
$\begin{quantikz}
    \lstick{$\ket{x_1}$}&\octrl{1}&\octrl{1}&\ctrl{1}&\ctrl{1}&\ctrl{1}&\qw\\
    \lstick{$\ket{x_2}$}&\ctrl{2}&\ctrl{1}&\octrl{1}&\ctrl{2}&\ctrl{1}&\qw\\
    \lstick{$\ket{y_1}$}&\qw&\targ{}&\targ{}&\qw&\targ{}&\qw\\
    \lstick{$\ket{y_2}$}&\targ{}&\octrl{-1}&\qw&\targ{}&\ctrl{-2}&\qw 
\end{quantikz}$
\subsection*{Exercise 4.37}
Identical to the 3x3 matrix we do the following,\\
$U_1=\begin{bmatrix}
    \frac{1}{\sqrt{2}}&\frac{1}{\sqrt{2}}&0&0\\
    \frac{1}{\sqrt{2}}&-\frac{1}{\sqrt{2}}&0&0\\
    0&0&1&0\\
    0&0&0&1
\end{bmatrix}$\\
$U_1U=\frac{1}{2}\begin{bmatrix}
    \sqrt{2}&\frac{1}{\sqrt{2}}(1+i)&0&\frac{1}{\sqrt{2}}(1-i)\\
    0&\frac{1}{\sqrt{2}}(1-i)&\sqrt{2}&\frac{1}{\sqrt{2}}(1+i)\\
    1&-1&1&-1\\
    1&-i&-1&i
\end{bmatrix}$\\
$U_2=\begin{bmatrix}
    \sqrt{\frac{2}{3}}&0&\frac{1}{\sqrt{3}}&0\\
    0&1&0&0\\
    \frac{1}{\sqrt{3}}&0&-\sqrt{\frac{2}{3}}&0\\
    0&0&0&1
\end{bmatrix}$\\
$U_2U_1U=\frac{1}{2}\begin{bmatrix}
    \sqrt{3}&\frac{i}{\sqrt{3}}&\frac{1}{\sqrt{3}}&-\frac{i}{\sqrt{3}}\\
    0&\frac{1}{\sqrt{2}}(1-i)&\sqrt{2}&\frac{1}{\sqrt{2}}(1+i)\\
    0&\frac{1}{\sqrt{2}}(\sqrt{3}+\frac{i}{\sqrt{3}})&-\sqrt{\frac{2}{3}}&\frac{1}{\sqrt{2}}(\sqrt{3}-\frac{i}{\sqrt{3}})\\
    1&-i&-1&i
\end{bmatrix}$\\
$U_3=\begin{bmatrix}
    \frac{\sqrt{3}}{2}&0&0&\frac{1}{2}\\
    0&1&0&0\\
    0&0&1&0\\
    \frac{1}{2}&0&0&-\frac{\sqrt{3}}{2}
\end{bmatrix}$\\
$U_3U_2U_1U=\frac{1}{2}\begin{bmatrix}
    2&0&0&0\\
    0&\frac{1}{\sqrt{2}}(1-i)&\sqrt{2}&\frac{1}{\sqrt{2}}(1+i)\\
    0&\frac{1}{\sqrt{2}}(\sqrt{3}+\frac{i}{\sqrt{3}})&-\sqrt{\frac{2}{3}}&\frac{1}{\sqrt{2}}(\sqrt{3}-\frac{i}{\sqrt{3}})\\
    0&i\frac{2}{\sqrt{3}}&\frac{2}{\sqrt{3}}&-i\frac{2}{\sqrt{3}}
\end{bmatrix}$\\
$U_4=\begin{bmatrix}
    1&0&0&0\\
    0&\frac{\sqrt{3}}{4}(1+i)&\frac{\sqrt{3}}{4}(\sqrt{3}-\frac{i}{\sqrt{3}})&0\\
    0&\frac{\sqrt{3}}{4}(\sqrt{3}+\frac{i}{\sqrt{3}})&-\frac{\sqrt{3}}{4}(1-i)&0\\
    0&0&0&1
\end{bmatrix}$\\
$U_4U_3U_2U_1U=\begin{bmatrix}
    1&0&0&0\\
    0&\sqrt{\frac{2}{3}}&i\frac{1}{\sqrt{6}}&\frac{1}{\sqrt{6}}\\
    0&0&\frac{1}{\sqrt{2}}&\frac{1}{\sqrt{2}}i\\
    0&\frac{1}{\sqrt{3}}i&\frac{1}{\sqrt{3}}&-\frac{1}{\sqrt{3}}i
\end{bmatrix}$\\
$U_5=\begin{bmatrix}
    1&0&0&0\\
    0&\sqrt{\frac{2}{3}}&0&-\frac{1}{\sqrt{3}}i\\
    0&0&1&0\\
    0&\frac{1}{\sqrt{3}}i&0&-\sqrt{\frac{2}{3}}
\end{bmatrix}$\\
$U_5U_4U_3U_2U_1U=\begin{bmatrix}
    1&0&0&0\\
    0&1&0&0\\
    0&0&\frac{1}{\sqrt{2}}&i\frac{1}{\sqrt{2}}\\
    0&0&-\frac{1}{\sqrt{2}}&i\frac{1}{\sqrt{2}}
\end{bmatrix}$\\
$U_6=\begin{bmatrix}
    1&0&0&0\\
    0&1&0&0\\
    0&0&\frac{1}{\sqrt{2}}&-\frac{1}{\sqrt{2}}\\
    0&0&-i\frac{1}{\sqrt{2}}&-i\frac{1}{\sqrt{2}}
\end{bmatrix}$\\
$U_6U_5U_4U_3U_2U_1U=I$\\
Therefore, $U=U_1^\dagger U_2^\dagger U_3^\dagger 
U_4^\dagger U_5^\dagger U_6^\dagger$.
\subsection*{Exercise 4.38}

\subsection*{Exercise 4.39}
$U$ acts non-trivially on $\ket{010}$ and $\ket{111}$.
Hence, from considering the Gray code $\ket{010}\rightarrow
\ket{011}\rightarrow\ket{111}$, we read off the circuit to be,\\
$\begin{quantikz}
    &\octrl{1}&\gate{\tilde{U}}&\octrl{1}&\qw\\
    &\ctrl{1}&\ctrl{-1}&\ctrl{1}&\qw\\
    &\targ{}&\ctrl{-1}&\targ{}&\qw
\end{quantikz}$
\subsection*{Exercise 4.40}
$E(R_{\hat{n}}(\alpha),R_{\hat{n}}(\alpha+\beta))=
E(R_{\hat{n}}(\alpha),R_{\hat{n}}(\beta)R_{\hat{n}}(\alpha))=
||(1-R_{\hat{n}}(\beta))R_{\hat{n}}(\alpha)\ket{\psi}||=
||(1-e^{-i\beta \hat{n}.\sigma/2})\ket{\psi}||=
\sqrt{\bra{\psi}(2-(e^{i\beta \hat{n}.\sigma/2}+
e^{-i\beta \hat{n}.\sigma/2}))\ket{\psi}}=
\sqrt{\bra{\psi}(2-2\cos{\frac{\beta}{2}})\ket{\psi}}=
\sqrt{2-2\cos{\frac{\beta}{2}}}=|1-e^{i\beta/2}|$\\
For a small $\beta$ such that $\alpha+\beta+m\theta=n\theta$ for $m<n$ we have,\\
$E(R_{\hat{n}}(\alpha),R_{\hat{n}}(\theta)^n)=
E(R_{\hat{n}}(\alpha),R_{\hat{n}}(\alpha)R_{\hat{n}}(\beta)R_{\hat{n}}(\theta)^m)=
E(I,R_{\hat{n}}(\beta)R_{\hat{n}}(\theta)^m)\leq E(I,R_{\hat{n}}(\beta))+
mE(I,R_{\hat{n}}(\theta)^m)=|1-e^{i\beta/2}|+m|1-e^{i\theta/2}|$\\
As $|\beta|,|\theta|<\delta$ we have that $|1-e^{i\beta/2}|+m|1-e^{i\theta/2}|<\frac{\epsilon}{3}$,
hence $E(R_{\hat{n}}(\alpha),R_{\hat{n}}(\theta)^n)<\frac{\epsilon}{3}$.
\subsection*{Exercise 4.41}
For an initial state $\ket{00\psi}$ the circuit performs the following,\\
$\ket{00\psi}\xrightarrow{H_1H_2}\frac{1}{2}(\ket{00}+\ket{01}+\ket{10}+\ket{11})\ket{\psi}
\xrightarrow{\text{Toffoli}}\frac{1}{2}(\ket{00\psi}+\ket{01\psi}+\ket{10\psi}+\ket{11(X\psi)})
\xrightarrow{S}\frac{1}{2}(\ket{00(S\psi)}+\ket{01(S\psi)}+\ket{10(S\psi)}+\ket{11(SX\psi)})
\xrightarrow{\text{Toffoli}}\frac{1}{2}(\ket{00(S\psi)}+\ket{01(S\psi)}+\ket{10(S\psi)}+\ket{11(XSX\psi)})
\xrightarrow{H_1H_2}\frac{1}{4}((\ket{0}+\ket{1})(\ket{0}+\ket{1})+(\ket{0}+\ket{1})(\ket{0}-\ket{1})+
(\ket{0}-\ket{1})(\ket{0}+\ket{1}))\ket{S\psi}+\frac{1}{4}(\ket{0}-\ket{1})(\ket{0}-\ket{1})\ket{XSX\psi}=
\frac{1}{4}(\ket{00}(3S+XSX)\ket{\psi}+(\ket{01}+\ket{10})(S-XSX)\ket{\psi}+\ket{11}(XSX-S)\ket{\psi})=
\frac{1}{4}(\ket{00}(3S+XSX)\ket{\psi}+(\ket{01}+\ket{10}-\ket{11})(S-XSX)\ket{\psi})$\\
If both measurements give $0$ then $\frac{1}{4}(3S+XSX)$ is applied to $\ket{\psi}$.\\
$\frac{1}{4}(3S+XSX)=\frac{1}{4}
\begin{bmatrix}
    3+i&0\\
    0&1+3i
\end{bmatrix}=\frac{1}{4}
\begin{bmatrix}
    i(1-3i)&0\\
    0&1+3i
\end{bmatrix}=
\frac{\sqrt{10}}{4}e^{i\pi/4}
\begin{bmatrix}
    e^{i\pi/4}e^{-i\arctan{3}}&0\\
    0&e^{-i\pi/4}e^{i\arctan{3}}
\end{bmatrix}=\frac{\sqrt{10}}{4}e^{\pi/4}R_Z(\theta)$,\\
as $\cos{\frac{\theta}{2}}\cos(\arctan{3}-\frac{\pi}{4})=\frac{2}{\sqrt{5}}$, hence
$\cos\theta=2\cos^2{\frac{\theta}{2}}-1=\frac{3}{5}$.\\
Otherwise, $Z$ is applied, as\\
$S-XSX=\begin{bmatrix}
    1-i&0\\
    0&i-1
\end{bmatrix}=e^{-i\pi/4}
\begin{bmatrix}
    1&0\\
    0&-1
\end{bmatrix}$\\
The final state is $\frac{1}{4}(\sqrt{10}e^{i\pi/4}\ket{00(R_Z(\theta)\psi)}
+e^{-i\pi/4}(\ket{01}+\ket{10}-\ket{11})Z\ket{\psi})$.\\
Therefore, $p(\ket{00})=\left|\frac{\sqrt{10}e^{i\pi/4}}{4}\right|=\frac{5}{8}$.\\
If anything other then $\ket{00}$ is measured we apply a $-Z$ gate if the measurement 
is $\ket{11}$ and $Z$ gate otherwise to the last qubit and afterwards apply the circuit again. Then for 
the overall circuit,\\
$p(R_Z(\theta))=\displaystyle \lim_{n \to +\infty}\sum_n\left(\frac{3}{8}\right)^{n-1}\frac{5}{8}=1$
\subsection*{Exercise 4.42}
1)As $\theta$ is rational $\exists m$ such that $m\theta=2\pi n$, hence raising both sides
to the power of this $m$ we have,\\
$e^{i2\pi n}=\frac{(3+4i)^m}{5^m}$\\
$1=\frac{(3+4i)^m}{5^m}$\\
$(3+4i)^m=5^m$\\
2)$3+4i=3+4i \mod{5}$\\
$(3+4i)^2=-7+24i=3+4i \mod{5}$\\
Let us show that if for some $a$ and $n$ we have $a=a\mod{n}$ and $a^2=a\mod{n}$ then
$a^m=a\mod{n}$. For $a^3$ we have $a^3=aa^2=anq_1+a^2=n(aq_1+q_2)+a=a\mod{n}$. Assume
true for $m-1$. Then $a^m=aa^{m-1}=anq_1+a^2=n(aq_1+q_2)+a=a\mod{n}$, hence by induction
it's true for all $m\in \mathbb{N}$. Therefore, we have that, $(3+4i)^m=3+4i\mod{5}$.
Hence, $(3+4i)^m$ is not a multiple of 5, therefore $\nexists m\in\mathbb{N}$ such that,
$(3+4i)^m=5^m$.
\subsection*{Exercise 4.43}
The circuit shown in Exercise 4.41 implements an $R_Z(\theta)$ gate with an 
irrational $\theta$ as shown in Exercise 4.42. $HR_Z(\theta)H=R_X(\theta)$, hence
we can convert between $R_{\hat{n}}(\theta)$ and $R_Z(\theta)$ as in Exercise 4.6 by only using
Hadamard, phase and Toffoli gates. Hence, in accordance with equation 4.76 we have
$E(R_Z(\alpha),R_Z(\theta)^n)=E(R_{\hat{n}}(\alpha),R_{\hat{n}}(\theta)^n)<\frac{\epsilon}{3}$.
Therefore, Hadamard, phase, CNOT and Toffoli gates are universal for quantum
computation.
\subsection*{Exercise 4.44}
\subsection*{Exercise 4.45}
\subsection*{Exercise 4.46}
$\rho$ is a complex $2^nx2^n$ hermitian matrix with trace 1. There are
$2*2^{2n}$ real components. The diagonal has to be real so we get 
$2*2^{2n}-2^n$. As the matrix is hermitian only half of the components
other then the diagonal elements are independent, hence the number of components
is $2*2^{2n}-2^n-(2*\frac{2^{2n}-2^n}{2})=4^n$. The trace being 1 fixes one 
of the diagonal elements, hence the number of independent real components is
$4^n-1$.
\subsection*{Exercise 4.47}
If $[A,B]=0$\\
$e^{A+B}=\displaystyle\sum_{n=0}^\infty\frac{(A+B)^n}{n!}=
\displaystyle\sum_{n=0}^\infty\sum_{k=0}^n{n\choose k}\frac{A^{n-k}B^k}{n!}=
\displaystyle\sum_{n=0}^\infty\sum_{k=0}^{n}\frac{A^{n-k}B^k}{(n-k)!k!}=
\displaystyle\sum_{k=0}^\infty\sum_{n=k}^\infty\frac{A^{n-k}B^k}{(n-k)!k!}=
\displaystyle\sum_{k=0}^\infty\sum_{n-k=0}^\infty\frac{A^{n-k}B^k}{(n-k)!k!}=
e^{A}e^{B}$\\
Hence, applying this to $e^{-iHt}$ we get the desired result.
\subsection*{Exercise 4.48}
$L\leq\displaystyle\sum_{i=0}^n{n\choose i}<c{n\choose c}=
c\displaystyle\frac{n!}{(n-c)!c!}=c\displaystyle\frac{n(n-1)\ldots(n-c+1)}{c!}=O(n^c)$
\subsection*{Exercise 4.49}
$e^{(A+B)\Delta t}=1+(A+B)\Delta t+\frac{1}{2}(A+B)^2\Delta t^2+O(\Delta t^3)=
1+(A+B)\Delta t+\frac{1}{2}(A^2+B^2+AB+BA)\Delta t^2+O(\Delta t^3)=
1+(A+B)\Delta t+\frac{1}{2}(A^2+B^2+2AB-(AB-BA))\Delta t^2+O(\Delta t^3)=
e^{A\Delta t}e^{B\Delta t}e^{-\frac{1}{2}[A,B]\Delta t^2}+O(\Delta t^3)$\\
$e^{i(A+B)\Delta t}=1+i(A+B)\Delta t +O(\Delta t^2)=e^{iA\Delta t}e^{iB\Delta t}+O(\Delta t^2)$\\
$e^{i(A+B)\Delta t}=1+i(A+B)\Delta t-\frac{1}{2}(A^2+B^2+AB+BA)\Delta t^2+
O(\Delta t^3)=1+i(\frac{1}{2}A+B+\frac{1}{2}A)\Delta t-(\frac{1}{4}A^2+\frac{1}{2}B^2+\frac{1}{4}AB+\frac{1}{4}BA)\Delta t^2+
O(\Delta t^3)=e^{iA\Delta t/2}e^{iB\Delta t}e^{A\Delta t/2}+O(\Delta t^3)$
\subsection*{Exercise 4.50}
a)From repeated use of 4.104 we have,\\
$U_{\Delta t}=[e^{-iH_1\Delta t}\ldots e^{-iH_L\Delta t}]
[e^{-iH_L\Delta t}\ldots e^{-iH_1\Delta t}]=
[e^{-iH_1\Delta t}\ldots e^{-iH_{L-1}\Delta t}]e^{-i2H_L\Delta t}
[e^{-iH_{L-1}\Delta t}\ldots e^{-iH_1\Delta t}]=
[e^{-iH_1\Delta t}\ldots e^{-iH_{L-2}\Delta t}]e^{-i2(H_{L-1}+H_L)\Delta t}
[e^{-iH_{L-2}\Delta t}\ldots e^{-iH_1\Delta t}]+O(\Delta t^3)=
e^{-i2H\Delta t}+O(\Delta t^3)$\\
b)$E(U_{\Delta t}^m, e^{-2miH\Delta t})\leq mE(U_{\Delta t}, e^{-2iH\Delta t})=
m||(U_{\Delta t}- e^{-2iH\Delta t})\ket{\psi}||=
m||O(\Delta t^3)\ket{\psi}||=m\alpha \Delta t^3$
\subsection*{Exercise 4.51}
$HXH=Z$ and $R_X(-\frac{\pi}{2})Y=Z$, hence the circuit is\\
$\begin{quantikz}
    &\gate{H}&\ctrl{3}&\qw&\qw&\qw&\qw&\qw&\ctrl{3}&\gate{H}&\qw\\
    &\gate{R_X(-\frac{\pi}{2})}&\qw&\ctrl{2}&\qw&\qw&\qw&\ctrl{2}&\qw&\gate{R_X(\frac{\pi}{2})}&\qw\\
    &\qw&\qw&\qw&\ctrl{1}&\qw&\ctrl{1}&\qw&\qw&\qw&\qw\\
    \lstick{$\ket{0}$}&\qw&\targ{}&\targ{}&\targ{}&\gate{e^{-i\Delta tZ}}&\targ{}&\targ{}&\targ{}&\qw&\qw
\end{quantikz}$

\end{document}