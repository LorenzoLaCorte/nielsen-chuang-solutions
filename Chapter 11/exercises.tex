\documentclass[a4paper,12pt]{article}
\usepackage{amsmath}
\usepackage[margin=0.9in]{geometry}
\usepackage{braket}
\begin{document}
\subsection*{Exercise 11.1}
For a fair coin $H(X)=-2\times\frac{1}{2}\log{\frac{1}{2}}=1$\\
For a fair die $H(X)=-6\times\frac{1}{6}\log{\frac{1}{6}}=1+\log{3}$\\
For an unfair coin we can write $H(X)=-p\log{p}-(1-p)\log{(1-p)}$ and 
for the unfair die $H(X)=-p_1\log{p_1}-p_2\log{p_2}-p_3\log{p_3}-p_4\log{p_4}-p_5\log{p_5}
-(1-p_1-p_2-p_3-p_4-p_5)\log{(1-p_1-p_2-p_3-p_4-p_5)}$.\\
Differentiating both of these we see that for both the global maxima is when 
all the probabilities are equal, therefore for the unfair coin and die the entropy will
decrease.
\subsection*{Exercise 11.2}
$I(p)=k\log{p}$ is a function of probability alone.\\
$\log{p}$ is smooth for $0<p\leq 1$\\
$I(pq)=k\log{(pq)}=k(\log{p}+\log{q})=I(p)+I(q)$
\subsection*{Exercise 11.3}
$H_{bin}(p)=-p\log{p}-(1-p)\log{(1-p)}$\\
$\displaystyle \frac{dH_{bin}}{dp}=-\frac{1}{\ln{2}}-\log{p}+\frac{1}{\ln{2}}+\log{(1-p)}=0$\\
$\displaystyle\frac{1-p}{p}=1$\\
Therefore, $p=\frac{1}{2}$.
\subsection*{Exercise 11.4}
For a function $f(x)$ to be concave we require $f^{\prime\prime}(x)<0$.\\
$\displaystyle \frac{d^2H_{bin}}{dp^2}=\frac{d}{dp}(\log{(1-p)}-\log{p})=
\frac{1}{\ln{2}(1-p)p}<0$\\
Hence, $H_{bin}$ is concave.
\subsection*{Exercise 11.5}
$H(p(x,y)||p(x)p(y))=\displaystyle\sum_{xy}p(x,y)\log{\frac{p(x,y)}{p(x)p(y)}}=
\sum_{xy}p(x,y)\log{p(x,y)}-\sum_{xy}p(x,y)\log{p(x)}-\sum_{xy}p(x,y)\log{p(y)}=
\sum_{xy}p(x,y)\log{p(x,y)}-\sum_{x}p(x)\log{p(x)}-\sum_{y}p(y)\log{p(y)}=
H(p(x))+H(p(y))-H(p(x,y))$\\
$H(p(x,y)||p(x)p(y))\geq 0$\\
Therefore,\\
$H(p(x))+H(p(y))-H(p(x,y))=H(X)+H(Y)-H(X,Y)\geq 0$\\
$H(X,Y)\leq H(X)+H(Y)$\\
If $X$ and $Y$ are independent then $p(x,y)=p(x)p(y)$. Therefore,\\
$H(X,Y)=-\displaystyle\sum_{xy}p(x,y)\log{p(x,y)}=
-\sum_{xy}p(x)p(y)\log{p(x)p(y)}=-\sum_{x}p(x)\log{p(x)}-\sum_{y}p(y)\log{p(y)}=H(X)+H(Y)$\\
Therefore, equality hold if and only if $X$ and $Y$ are independent.
\subsection*{Exercise 11.6}

\subsection*{Exercise 11.7}
\subsection*{Exercise 11.8}
\subsection*{Exercise 11.9}
\subsection*{Exercise 11.10}
\subsection*{Exercise 11.12}
\subsection*{Exercise 11.13}
\subsection*{Exercise 11.14}
\subsection*{Exercise 11.15}
\subsection*{Exercise 11.16}
\subsection*{Exercise 11.17}
\subsection*{Exercise 11.18}
\subsection*{Exercise 11.19}
\subsection*{Exercise 11.20}
\subsection*{Exercise 11.21}
\subsection*{Exercise 11.22}
\subsection*{Exercise 11.23}
\subsection*{Exercise 11.24}
\subsection*{Exercise 11.25}
\subsection*{Exercise 11.26}
\end{document}