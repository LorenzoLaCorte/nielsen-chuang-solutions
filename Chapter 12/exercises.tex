\documentclass[a4paper,12pt]{article}
\usepackage{amsmath}
\usepackage[margin=0.9in]{geometry}
\usepackage{braket}
\usepackage{graphicx}
\usepackage{quantikz}
\begin{document}
\subsection*{Exercise 12.1}
Let $\ket{\psi}=a\ket{0}+b\ket{1}$. Then, $\ket{\varphi}=b\ket{0}-a\ket{1}$.\\
We require a gate which transforms between the basis of $\ket{\psi}$ and $\ket{\varphi}$,
and the computational basis. This can be performed using the unitary,
$U=U^dagger=\begin{bmatrix}
    a&b\\
    b&-a
\end{bmatrix}$. Using $U$ we construct the following circuit which performs the cloning.\\
$\begin{quantikz}
    \lstick{$\ket{\psi}$}&\gate{U}&\ctrl{1}&\gate{U}&\qw&\rstick{$\ket{\psi}$}\\
    \lstick{$\ket{0}$}&\qw&\targ{}&\gate{U}&\qw&\rstick{$\ket{\psi}$}
\end{quantikz}$
\subsection*{Exercise 12.2}
$\displaystyle\sum_y\sqrt{E_y}\otimes U_y(\sigma \otimes \ket{0}\bra{0})\sqrt{E_y}\otimes U_y=
\sum_y\sqrt{E_y}\sigma\sqrt{E_y}\otimes \ket{0+y}\bra{0+y}=
\sum_y\sqrt{E_y}\sigma\sqrt{E_y}\otimes \ket{y}\bra{y}$
\subsection*{Exercise 12.3}
From the Holevo bound $H(X:Y)\leq S(\rho)$, however $S(\rho)$ is bounded by $\log{2^n}=n$, 
hence no more than $n$ bits can be transferred.
\subsection*{Exercise 12.4}
Let $\rho_i=\ket{X_i}\bra{X_i}$, and  $\rho=\frac{1}{4}\sum_i\rho_i=\frac{I}{2}$, which
is the maximally mixed state, hence $S(\rho)=1$ and the $S(\rho_i)=0$ as the $\rho_i$ are
pure states. Then,\\
$\chi=S(\rho)-\displaystyle\sum_ip_iS(\rho_i)=1$\\
We have tetrahedral symmetry for the states $\ket{X_i}$, hence we consider POVMs with 
the same symmetry. Consider $E_i=\frac{1}{2}\ket{X_i}\bra{X_i}$, and $F_i=\frac{1}{2}\ket{\psi_i}\bra{\psi_i}$,
where\\
$\ket{\psi_1}=\ket{1}$\\
$\ket{\psi_2}=\sqrt{\frac{1}{3}}(\sqrt{2}e^{\pi i/3}\ket{0}+\ket{1})$\\
$\ket{\psi_3}=\sqrt{\frac{1}{3}}(\sqrt{2}e^{\pi i}\ket{0}+\ket{1})$\\
$\ket{\psi_4}=\sqrt{\frac{1}{3}}(\sqrt{2}e^{5\pi i/3}\ket{0}+\ket{1})$\\
Using $H(X:Y)=H(Y)-H(Y|X)$, $P(X=x)=\frac{1}{4}$, $p(Y=y|x=x)=tr(\rho_xE_y)$ and
Bayes theorem we get,\\
$H(Y)=-\displaystyle\sum_yp(Y=y)\log{p(Y=y)}=\\
-\sum_y\left(\sum_xp(X=x)p(Y=y|X=x)\right)\log{\left(\sum_xp(X=x)p(Y=y|X=x)\right)}$\\
$H(Y|X)=\displaystyle - \sum_xp(X=x)\sum_yp(Y=y|X=x)\log{p(Y=y|X=x)}$\\
The calculation can be found in 12.3.py and gives $H(X:Y)_E\approx 0.208$ and
$H(X:Y)_F\approx 0.415$. Therefore, ${F_i}$ is the desired POVM.\\
As shown in DOI: 10.1109/TIT.1978.1055941 the maximum information is $\log{\frac{4}{3}}\approx 0.415$.
\subsection*{Exercise 12.5}
We know that $p(x\in T(n,\epsilon))\geq 1-\delta$, hence $1-p(x\in T(n,\epsilon))\leq\delta$. 
The expected number of bits required for a single symbol when 
the $x$s are $\epsilon$-typical will be,\\
$E(n,\epsilon)=H(X)p(x\in T(n,\epsilon))+\frac{\log{d^n}}{n}p(x\notin T(n,\epsilon))\leq\\
Rp(x\in T(n,\epsilon))+\log{d}(1-p(x\in T(n,\epsilon)))\leq R+\delta\log{\delta}$\\
Hence,\\
$E(n,\epsilon)-R\leq\delta\log{d}$
\subsection*{Exercise 12.6}
$C_X=p^{n-wt(X)}(1-p)^{wt(X)}$\\
Where, $wt(X)$ is the Hamming weight.
\subsection*{Exercise 12.7}
This is the circuit outlined in Box 12.4 but with the basis change gate $V$ set to $I$.
\subsection*{Exercise 12.8}
Taking $p_J=p_{j_1}p_{j_2}\ldots p_{j_n}$ and using Exercise 9.20 for the concavity of fidelity,
 from Equation 9.141 we have,\\
$\bar{F}\geq \displaystyle F\left(\sum_Jp_J\rho_J, D^n \circ C^n\right)=
F\left(\sum_Jp_{j_1}\ket{\psi_{j_1}}\bra{\psi_{j_1}}\otimes p_{j_2}\ket{\psi_{j_2}}\bra{\psi_{j_2}}
\ldots p_{j_n}\ket{\psi_{j_n}}\bra{\psi_{j_n}}, D^n \circ C^n\right)=
F(\rho^{\otimes n},D^n \circ C^n)=F(\rho, D^n \circ C^n)^n\geq (1-2\delta)^n$\\
Where the last inequality is only true for $R>S(\rho)$.
\subsection*{Exercise 12.9}

\subsection*{Exercise 12.10}
\subsection*{Exercise 12.11}
\subsection*{Exercise 12.12}
\subsection*{Exercise 12.13}
\subsection*{Exercise 12.14}
\subsection*{Exercise 12.15}
\subsection*{Exercise 12.16}
\subsection*{Exercise 12.17}
\subsection*{Exercise 12.18}
\subsection*{Exercise 12.19}
\subsection*{Exercise 12.20}

\end{document}