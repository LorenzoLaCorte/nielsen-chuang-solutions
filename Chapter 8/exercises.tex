\documentclass[a4paper,12pt]{article}
\usepackage{amsmath}
\usepackage[margin=0.9in]{geometry}
\usepackage{braket}
\usepackage{graphicx}
\begin{document}
\subsection*{Exercise 8.1}
Under the transformation $\rho \rightarrow \mathcal{E}(\rho)$, the state transforms as
$\ket{\psi}\rightarrow U\ket{\psi}$. Hence, the new density operator is 
$\rho^\prime =U\ket{\psi}\bra{\psi}U^\dagger=U\rho U^\dagger$, and therefore $\rho$ transforms as
$\rho \rightarrow U\rho U^\dagger$.
\subsection*{Exercise 8.2}
Let $\rho =\displaystyle \sum_i p_i \ket{i}\bra{i}$, hence after the measurement
for each of the $i$ states will take the form, $\ket{i^\prime}=\frac{M_m\ket{i}}{\sqrt{\bra{i}
M_m^\dagger M_m\ket{i}}}$. Therefore, for the final state $\rho^\prime$ we'll have,\\
$\rho^\prime =\displaystyle \sum_i p_i \frac{M_m\ket{i}\bra{i}M_m^\dagger}{\sqrt{\bra{i}
M_m^\dagger M_m\ket{i}}\sqrt{\bra{i}
M_m M_m^\dagger\ket{i}}}=\frac{\mathcal{E}_m(\rho)}{tr(\mathcal{E}_m(\rho))}$\\
For the probability of the $m$ state, using $p(m|i)=\bra{i}M_m^\dagger M_m\ket{i}$, we get\\
$p(m)=\displaystyle \sum_{i} p_i p(m|i)=
\displaystyle \sum_{i} p_i \bra{i}M_m^\dagger M_m\ket{i}=
\displaystyle \sum_{i} p_i tr(M_m^\dagger M_m\ket{i}\bra{i})=
tr(\mathcal{E}_m(\rho))$
\subsection*{Exercise 8.3}
Initially we have the state $\rho\otimes \ket{0_{CD}}\bra{0_{CD}}$. Consider the action
of $\mathcal{E}$($i$ basis for $A$, $j$ basis for $D$),\\
$\mathcal{E}(\rho)=tr_A(tr_D(U[\rho\otimes \ket{0_{CD}}\bra{0_{CD}}]U^\dagger))=
\displaystyle \sum_{i} \sum_{j} \bra{i}\bra{j}U[\rho\otimes \ket{0_{CD}}\bra{0_{CD}}]U^\dagger
\ket{j}\ket{i}=\\
\displaystyle \sum_{i} \sum_{j} \bra{i}\bra{j}U\ket{0_{CD}}\rho\bra{0_{CD}}U^\dagger
\ket{j}\ket{i}=\displaystyle \sum_j E_j\rho E_j^\dagger$.\\
where $E_j=\displaystyle \sum_i  \bra{i}\bra{j}U\ket{0_{CD}}$\\
Also, (using $\displaystyle \sum_{i} \ket{i}\bra{i}=I$)\\
$\displaystyle \sum_j E_j^\dagger E_j=\displaystyle \sum_i\sum_j 
\bra{0_{CD}}U^\dagger \ket{j}\ket{i}\bra{i}\bra{j}U\ket{0_{CD}}=
I\bra{0_{CD}}U^\dagger U\ket{0_{CD}} =I\braket{0_{CD}|0_{CD}}=I$\\
\subsection*{Exercise 8.4}
$E_k=\bra{k}U\ket{0}$, hence using the orthogonality of the $\ket{0}$ and $\ket{1}$
states, $E_0=P_0$, $E_1=P_1$. Therefore,\\
$\mathcal{E}(\rho)=\ket{0}\bra{0}\rho \ket{0}\bra{0}+\ket{1}\bra{1}\rho \ket{1}\bra{1}$
\subsection*{Exercise 8.5}
$E_0=\frac{X}{\sqrt{2}}$, $E_1=\frac{Y}{\sqrt{2}}$\\
$\mathcal{E}(\rho)=\frac{1}{2}(X\rho X^\dagger + Y\rho Y^\dagger)=
\frac{1}{2}(X\rho X - Y\rho Y)$
\newpage
\subsection*{Exercise 8.6}
In general the composition of quantum operations is still a quantum operation, hence we
only prove the general case.\\
Let $\rho$ belong to a Hilbert Space $\mathcal{H}$ and let the quantum operations be given by,
$\mathcal{E}(\rho)=\displaystyle \sum_i E_i \rho E_i^\dagger$ and
$\mathcal{F}(\rho)=\displaystyle \sum_i F_i \rho F_i^\dagger$.\\
As by definition, $\mathcal{E}$ and $\mathcal{F}$ are quantum operations, there exist states
$\omega_\mathcal{E}$ and $\omega_\mathcal{F}$ and unitary operators
$U_\mathcal{E}$ and $U_\mathcal{F}$ on Hilbert spaces
$\mathcal{K}_\mathcal{E}$ and $\mathcal{K}_\mathcal{F}$, respectively, such that\\
$\mathcal{E}(\rho)=tr_{\mathcal{K}_\mathcal{E}}(U_\mathcal{E}[\rho\otimes\omega_\mathcal{E}]U_\mathcal{E}^\dagger)$ and
$\mathcal{F}(\rho)=tr_{\mathcal{K}_\mathcal{F}}(U_\mathcal{F}[\rho\otimes\omega_\mathcal{F}]U_\mathcal{F}^\dagger)$.\\
Consider the Hilbert space $\mathcal{K}=\mathcal{K}_\mathcal{E}\otimes\mathcal{K}_\mathcal{F}$ and the state
$\omega=\omega_\mathcal{E}\otimes\omega_\mathcal{F}$. Consider the ampliations
$\hat{U}_\mathcal{E}$ and $\hat{U}_\mathcal{F}$ of $U_\mathcal{E}$ and $U_\mathcal{F}$ to
$\mathcal{H}\otimes\mathcal{K}$, i.e $\hat{U}_\mathcal{E}=U_\mathcal{E}\otimes \mathcal{I}$ and
$\hat{U}_\mathcal{F}=\mathcal{I}\otimes U_\mathcal{F}$. Lastly, take $U=\hat{U}_\mathcal{F} \hat{U}_\mathcal{E}$,
which is an operator on $\mathcal{H}\otimes\mathcal{K}$. Finally, consider\\
\begin{align*}
tr_\mathcal{K}(U[\rho\otimes\omega]U^\dagger)&=
tr_{\mathcal{K}_\mathcal{E}\otimes\mathcal{K}_\mathcal{F}}
(\hat{U}_\mathcal{F}\hat{U}_\mathcal{E}[\rho\otimes\omega_\mathcal{E}\otimes\omega_\mathcal{F}]
\hat{U}_\mathcal{E}\hat{U}_\mathcal{F})\\ 
&=tr_{\mathcal{K}_\mathcal{F}}(tr_{\mathcal{K}_\mathcal{E}}(\hat{U}_\mathcal{F}
(U_\mathcal{E}[\rho\otimes\omega_\mathcal{E}]U_\mathcal{E}^\dagger\otimes\omega_\mathcal{F})\hat{U}_\mathcal{F}^\dagger))\\
&= tr_{\mathcal{K}_\mathcal{F}}(U_\mathcal{F}(tr_{\mathcal{K}_\mathcal{E}}
(U_\mathcal{E}[\rho\otimes\omega_\mathcal{E}]U_\mathcal{E}^\dagger)\otimes\omega_\mathcal{F})U_\mathcal{F}^\dagger)\\
&= tr_{\mathcal{K}_\mathcal{F}}(U_\mathcal{F}(\mathcal{E}(\rho)
\otimes\omega_\mathcal{F})U_\mathcal{F}^\dagger)\\
&=\mathcal{F}(\mathcal{E}(\rho))
\end{align*}
From the trace as previously we can obtain an operator-sum representation, hence
the composition even for different input and output spaces is a quantum operation.
\subsection*{Exercise 8.7}
Again consider, $\rho^{QE}=\rho\otimes\sigma$. The final state after a general measurement
with outcome $m$ is,\\
$\frac{M_mU(\rho\otimes\sigma)U^\dagger M_m^\dagger}{tr(M_mU(\rho\otimes\sigma)U^\dagger M_m^\dagger)}$\\
Hence, tracing out $E$ the final state of Q is,\\
$\frac{tr_E(M_mU(\rho\otimes\sigma)U^\dagger M_m^\dagger)}{tr(M_mU(\rho\otimes\sigma)U^\dagger M_m^\dagger)}$\\
Define, $\mathcal(E)_m(\rho)=tr_E(M_mU(\rho\otimes\sigma)U^\dagger M_m^\dagger)$. Let 
$\sigma=\displaystyle \sum_J\ket{j}\bra{j}$ and consider an orthonormal basis $\ket{e_k}$
for the system $E$. We get,\\
$\mathcal{E}_m(\rho)=\displaystyle \sum_{jk}q_jtr_E(\ket{e_k}\bra{e_k}M_mU(\rho\otimes\sigma)U^\dagger M_m^\dagger\ket{e_k}\bra{e_k})=
\displaystyle \sum_{jk} E_{jk}\rho E_{jk}^\dagger$\\
where $E_{jk}=\sqrt{q_j}\bra{e_k}M_mU\ket{j}$
\subsection*{Exercise 8.8}
The process will be identical to the trace-preserving method, with the addition of the 
$E_\infty$ operation element. Additionally, we need to add another orthonormal basis vector
$\ket{e_\infty}$ to our basis, i.e ampliate the Hilbert Space of the environment.
\newpage
\subsection*{Exercise 8.9}
Consider the action of $U$ on $\rho\otimes \ket{e_0}\bra{e_0}$ succeeded by a measurement by $P_m$.
Tracing over this will give the probability of the outcome $m$.\\
\begin{align*}
    p(m)&=tr(P_m U(\rho\otimes \ket{e_0}\bra{e_0})U^\dagger P_m)\\
&=tr(\displaystyle \sum_k \ket{m,k}\bra{m,k}U\ket{e_0}\rho\bra{e_0}U^\dagger \ket{m,k}\bra{m,k})\\
&=tr(\displaystyle \sum_{k,m^\prime,k^\prime} \ket{m,k}\bra{m,k}E_{m^\prime k^\prime}\ket{m^\prime, k^\prime}
\rho\bra{m^\prime, k^\prime}E_{m^\prime k^\prime}^\dagger \ket{m,k}\bra{m,k})\\
&=tr(\displaystyle \sum_{k} \ket{m,k}E_{m k}
\rho E_{mk}^\dagger\bra{m,k})\\
&=tr_Q(tr_E(\displaystyle \sum_{k} \ket{m,k}E_{mk}
\rho E_{mk}^\dagger\bra{m,k}))\\
&=tr_Q(\displaystyle \sum_{k} E_{mk}
\rho E_{mk}^\dagger)\\
&=tr_Q(\mathcal{E}_m(\rho))=tr(\mathcal{E}_m(\rho))
\end{align*}\\

For the state we have,
$\frac{tr_E(P_m U(\rho\otimes \ket{e_0}\bra{e_0})U^\dagger P_m)}{p(m)}=\frac{\mathcal{E}_m(\rho)}{tr(\mathcal{E}_m(\rho))}$
\subsection*{Exercise 8.10}
\subsection*{Exercise 8.15}
This is the bit flip channel with $p=0.5$, hence it deforms into a line on the x-axis.
\subsection*{Exercise 8.16}
\subsection*{Exercise 8.17}
$\mathcal{E}(I)=\frac{I+XX+YY+ZZ}{4}=\frac{4I}{4}=I$\\
$\mathcal{E}(X)=\frac{X+XXX+YXY+ZXZ}{4}=\frac{X+X-X-X}{4}=0$\\
Similarly, $\mathcal{E}(Y)=\mathcal{E}(Z)=0$\\
$\rho=\frac{I+\vec{r}\cdot\vec{\sigma}}{2}\\
2\rho=I+r_xX+r_yY+r_zZ$\\
Left and right multiplying by $X$, $Y$ and $Z$ we get,\\
$2X\rho X=I+r_xX-r_yY-r_zZ\\
2Y\rho Y=I-r_xX+r_yY-r_zZ\\
2Z\rho Z=I-r_xX-r_yY+r_zZ$\\
Adding all 4 equations,\\
$2(\rho+X\rho X+Y\rho Y+Z\rho Z)=4I\\
\frac{I}{2}=\frac{\rho+X\rho X+Y\rho Y+Z\rho Z}{4}$
\subsection*{Exercise 8.18}
We have, $\mathcal{E}(\rho)=\rho^\prime=\displaystyle \frac{pI}{2}+(1-p)\rho$ and
$\rho=\displaystyle\frac{I+r\cdot\sigma}{2}$. Hence substituting $\rho$ we get,\\
$\rho^\prime=\displaystyle \frac{I}{2}+\frac{(1-p)r\cdot\sigma}{2}$.
Now we need to find the eigenvalues of $\rho$ and $\rho^\prime$. For $\rho^\prime$ we have,\\
$\begin{vmatrix}
    \frac{1}{2}+\frac{1-p}{2}r_z-\lambda & \frac{1-p}{2}(r_x-ir_y)\\
    \frac{1-p}{2}(r_x+ir_y) &      \frac{1}{2}-\frac{1-p}{2}r_z-\lambda
\end{vmatrix}
=0$\\
Hence, $\lambda=1\pm\frac{1-p}{4}|r|$ and similarly for $\rho$, $\lambda=1\pm\frac{1}{4}|r|$.\\
Therefore we have,\\
$tr(\rho)=(1-\frac{|r|}{4})^k+(1+\frac{|r|}{4})^k=\displaystyle \sum_{n}{k\choose 2n}\left(\frac{|r|}{4}\right)^{2n}$\\
$tr(\rho^\prime)=(1-\frac{(1-p)|r|}{4})^k+(1+\frac{(1-p)|r|}{4})^k=\displaystyle \sum_{n}{k\choose 2n}\left(\frac{(1-p)|r|}{4}\right)^{2n}$\\
Therefore, $tr(\rho^\prime)\leq tr(\rho)$ with equality for $p=0$.

\subsection*{Exercise 8.19}
We have $\mathcal{E}(\rho)=\frac{pI}{d}+(1-p)\rho$. We know that $tr(\rho)=1$, hence can write,
$\frac{I}{d}=\frac{I}{d}tr(\rho)$. Consider an orthonormal basis $\ket{i}$ for the system. This gives,\\
$\cfrac{I}{d}=\displaystyle \cfrac{1}{d}\sum_i \ket{i}\bra{i}\sum_j \bra{j}\rho\ket{j}=
\displaystyle \frac{1}{d}\sum_{i,j} \ket{i} \bra{j}\rho\ket{j}\bra{i}$\\
Hence, we can choose as the operation elements $\{\sqrt{\frac{p}{d}}\ket{i}\bra{j}\}$
\subsection*{Exercise 8.20}
Let the initial state be $\ket{\psi_0}=a\ket{00}+b\ket{10}$. Then applying the controlled-$R_y$ and CNOT gates we get.\\
After the $R_y$ we have,\\
$\ket{\psi_1}=a\ket{00}+b\cos{\frac{\theta}{2}}\ket{10}+b\sin{\frac{\theta}{2}}\ket{11}$\\
After the CNOT we have,\\
$\ket{\psi_2}=a\ket{00}+b\cos{\frac{\theta}{2}}\ket{10}+b\sin{\frac{\theta}{2}}\ket{01}$\\
Tracing over the environment we get,\\
$tr_E(\ket{\psi_2}\bra{\psi_2})=(a\ket{0}+b\cos{\frac{\theta}{2}}\ket{1})(a^*\bra{0}+b^*\cos{\frac{\theta}{2}}\bra{1})+
bb^*\sin^2{\frac{\theta}{2}}\ket{0}\bra{0}=
\begin{bmatrix}
    |a|^2+|b|^2\sin^2{\frac{\theta}{2}} & ab^*\cos{\frac{\theta}{2}}\\
    ba^*\cos{\frac{\theta}{2}} &    |b|^2\cos^2{\frac{\theta}{2}}
\end{bmatrix}$\\
If we apply amplitude damping to our original state we get,\\
$\mathcal{E}_{AD}=E_0
\begin{bmatrix}
    |a|^2&ab^*\\
    ba^*& |b|^2
\end{bmatrix}
E_0^\dagger +E_1
\begin{bmatrix}
    |a|^2&ab^*\\
    ba^*& |b|^2
\end{bmatrix}
E_1^\dagger=
\begin{bmatrix}
    |a|^2+\gamma|b|^2& ab*\sqrt{1-\gamma}\\
    ba^*\sqrt{1-\gamma} & |b|^2(1-\gamma)
\end{bmatrix}$\\
Comparing with the model above we see that, the circuit does indeed model the quantum operation
with $\gamma=\sin^2{\frac{\theta}{2}}$.

\end{document}