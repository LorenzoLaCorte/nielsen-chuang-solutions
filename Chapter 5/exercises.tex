\documentclass[a4paper,12pt]{article}
\usepackage{amsmath}
\usepackage[margin=0.9in]{geometry}
\usepackage{braket}
\usepackage{graphicx}
\usepackage{tikz}
\usepackage{amssymb}
\usepackage{mathtools}
\usetikzlibrary{quantikz}
\begin{document}
\subsection*{Exercise 5.1}
$U\ket{j}=\frac{1}{\sqrt{N}}\displaystyle\sum_{k=0}^{N-1}e^{2\pi ijk/N}\ket{k}$\\
$\bra{j^\prime}U^\dagger U\ket{j}=\frac{1}{N}\displaystyle
\sum_{k^\prime=0}^{N-1}\sum_{k=0}^{N-1}e^{-2\pi ij^\prime k^\prime /N}e^{2\pi ijk/N}\delta_{k,k^\prime}=
\frac{1}{N}\displaystyle
\sum_{k=0}^{N-1}e^{2\pi i(j-j^\prime)k/N}=
\frac{1}{N}N\delta_{j,j^\prime}=\delta_{j,j^\prime}$\\
Therefore, $U^\dagger U =I$, hence $U$ is unitary.
\subsection*{Exercise 5.2}
$\ket{00\ldots 0}=\frac{1}{\sqrt{N}}
\displaystyle\sum_{k=0}^{N-1}\ket{k}=\frac{1}{2^{n/2}}
\displaystyle\sum_{x_i\in\{0,1\}}\ket{x_1x_2\ldots x_n}$
\subsection*{Exercise 5.3}
For each $y_k$ we perform $2^n$ additions and there are $2^n$ $y_k$ to
calculate, hence in total we require $\Theta(2^{2n})$ operations.\\
(Cooley-Turkey Algorithm) For each $x_k$ we can separate the sum into odd and even 
indices, then we require $2^n$ operations assuming the two separate sums are known.
This can be done recursively, splitting each sum into 2 pieces. This leads to the number
of operations to be $\Theta(2^n\log{2^n})=\Theta(n2^n)$.
\subsection*{Exercise 5.4}
Let $R_k=e^{i\alpha}AXBXC$ with $ABC=I$. Taking $\alpha=\frac{\pi i}{2^k}$, $A=I$
$B=R_Z(-\frac{\pi i}{2^{k}})$ and $C=R_Z(\frac{\pi i}{2^{k}})$ we see that $ABC=I$ and
$AXBXC=XR_Z(-\frac{\pi i}{2^{k}})XR_Z(\frac{\pi i}{2^{k}})=
XXR_Z(\frac{\pi i}{2^{k}})R_Z(\frac{\pi i}{2^{k}})=
R_Z(\frac{2\pi i}{2^{k}})$. Hence, the circuit will be,\\
$\begin{quantikz}
    &\ctrl{1}&\qw&\ctrl{1}&\phase{e^{\pi i/2^k}}&\qw\\
    &\targ{}&\gate{R_Z(-\frac{\pi i}{2^{k}})}&\targ{}&\gate{R_Z(\frac{\pi i}{2^{k}})}&\qw
\end{quantikz}$

\subsection*{Exercise 5.5}
$FT^{-1}=FT^\dagger$
\subsection*{Exercise 5.6}
In the circuit we have $m=\frac{n(n+1)}{2}=\Theta(n^2)$ $R_k$ gates. Using the result of Box 4.1,\\
$E(U,V)\leq m\frac{1}{p(n)}=\Theta(\frac{n^2}{p(n)})$
\subsection*{Exercise 5.7}
Let $\ket{j}=\ket{j_0j_2\ldots j_{n-1}}$, then the circuit implements the following,\\
$\ket{j}\ket{u}\rightarrow \ket{j}((U^{2^0})^{j_0}(U^{2^1})^{j_1}\ldots(U^{2^{n-1}})^{j_{n-1}})\ket{u}=
\ket{j}U^{j_02^0+j_12^1+\ldots+j_{n-1}2^{n-1}}\ket{u}=\ket{j}U^j\ket{u}$
\subsection*{Exercise 5.8}
With probability $|c_u|^2$ we will be measuring $\varphi_u$ for the state $\ket{u}$.
If $t$ is of the form of 5.35 each $\tilde{\varphi}_u$ is accurate to $n$ bits of $\varphi_u$ with
probability $1-\epsilon$. Hence, the total probability of measuring $\varphi_u$ accurate to $n$ bits
is $|c_u|^2(1-\epsilon)$.
\subsection*{Exercise 5.9}
For this $U$ $\varphi_0=0$ and $\varphi_1=\frac{1}{2}$, hence the circuit is,\\
$\begin{quantikz}
    \lstick{$\ket{0}$}&\gate{H}&\ctrl{1}&\gate{FT^\dagger}&\meter{}\\
    \lstick{$\ket{u}$}&\qw&\gate{U}&\qw&\qw
\end{quantikz}$
The state before the measurement is $\ket{0}\ket{u_0}-\ket{1}\ket{u_1}$, hence after the
measurement it will collapse into the $+1$ or $-1$ eigenbasis. For a first register with a
single qubit $FT^\dagger=H$, hence this is the same circuit as that in Exercise 4.34.
\subsection*{Exercise 5.10}
$5=5\bmod 21$, $5^2=4\bmod 21$,$5^3=20\bmod 21$, $5^4=16\bmod 21$, $5^5=17\bmod 21$ and
$5^6=1\bmod 21$. Hence, the order is $6$. 
\subsection*{Exercise 5.11}
As $gcd(x, N)=1$, from Euler's formula $x^{\varphi(N)}=1\bmod N$. $\varphi(N)$ is the number
of $y$ such that $gcd(y,N)=1$ and $y<N$, hence $\varphi(N)<N$. Therefore, there always exists a
number $r\leq N$, such that $x^r=1(\bmod N)$.
\subsection*{Exercise 5.12}
$\bra{y^\prime}U^\dagger U\ket{y}=\braket{xy^\prime|xy}=\braket{y^\prime|y}\bmod N$\\
$0\leq y\leq N-1$, hence $\braket{y^\prime|y}\bmod N=\braket{y^\prime|y}=\delta_{y,y^\prime}$. Therefore,
$\bra{y^\prime}U^\dagger U\ket{y}=\delta_{y,y^\prime}$.
Hence, $U$ is unitary.
\subsection*{Exercise 5.13}
$\displaystyle\frac{1}{\sqrt{r}}\displaystyle\sum_{s=0}^{r-1}\ket{u_s}=
\frac{1}{r}\displaystyle\sum_{s=0}^{r-1}\sum_{k=0}^{r-1}e^{-2\pi isk/r}\ket{x^k\bmod N}=
\frac{1}{r}\displaystyle\sum_{k=0}^{r-1}\sum_{s=0}^{r-1}e^{-2\pi isk/r}\ket{x^k\bmod N}=\\
=\frac{1}{r}\displaystyle\sum_{k=0}^{r-1}r\delta_{k0}\ket{x^k\bmod N}=\ket{1}$\\
$\displaystyle\frac{1}{\sqrt{r}}\displaystyle\sum_{s=0}^{r-1}e^{2\pi isk/r}\ket{u_s}=
\frac{1}{r}\displaystyle\sum_{s=0}^{r-1}\sum_{k^\prime=0}^{r-1}
e^{2\pi is(k-k^\prime)/r}\ket{x^{k^\prime}\bmod N}=
\frac{1}{r}\displaystyle\sum_{k^\prime=0}^{r-1}r\delta_{k,k^\prime}\ket{x^{k^\prime}\bmod N}=
\ket{x^k\bmod N}$
\subsection*{Exercise 5.14}
For $V$,\\
$\ket{\psi}=\displaystyle\sum_{j=0}^{2^t-1}\ket{j}V^j\ket{0}=
\displaystyle\sum_{j=0}^{2^t-1}\ket{j}\ket{0+x^j\bmod N}=
\displaystyle\sum_{j=0}^{2^t-1}\ket{j}\ket{x^j\bmod N}$\\
Writing $x^j(\bmod N)=(x^{j_t{2^{t-1}}}(\bmod N))(x^{j_{t-1}{2^{t-2}}}(\bmod N))
\ldots (x^{j_1{2^0}}(\bmod N))$, each modular multiplication requires $O(L^2)$ gates, hence
for the total product of $t-1$ modular multiplications we require $O(L^3)$ gates, and uses the circuit
shown in figure 5.2. The addition of $k$ is done after the modular 
multiplications and requires $O(L)$ gates, hence in total we still require $O(L^3)$ gates.
\subsection*{Exercise 5.15}
Let $m=[x,y]$ be the lowest common multiple. Let $M$ be any common multiple. Then
we can write $M=mq+r$. $x$ and $y$ divide both $M$ and $m$, hence they also divide $r$, 
meaning it's a common multiple, but $r<m$ and $m$ is the lowest common multiple, therefore
$r=0$. Now let $x=(x,y)x_1$ and $y=(x,y)y_1$ with $(x_1,y_1)=1$. $x$ and $y$ divide
$(x,y)x_1y_1$ hence it's a common multiple, therefore we can write $(x,y)x_1y_1=mq_1$. Therefore, we have
$x_1=\frac{m}{y}q_1$ and $y_1=\frac{m}{x}q_1$, hence $q_1$ divides both $x_1$ and $y_1$.
However, $(x_1,y_1)=1$, hence $q_1=1$. Hence, $[x,y]=(x,y)x_1y_1=(x,y)x_1(x,y)y_1/(x,y)=
xy/(x,y)$.\\
We can use Stein's gcd algorithm which requires $O(L^2)$ gates.
\subsection*{Exercise 5.16}
$\displaystyle\int_x^{x+1}\frac{1}{y^2}dy=\frac{1}{x(x+1)}$\\
Consider $\displaystyle\frac{1}{x(x+1)}-\frac{2}{3x^2}=\frac{x-1}{3x^2(x+1)}$\\
For $x\geq 2$ this is always greater than $0$, hence 
$\displaystyle\int_x^{x+1}\frac{1}{y^2}dy\geq\frac{2}{3x^2}$.\\
$\displaystyle\frac{3}{4}=\frac{3}{2}\displaystyle\int_2^\infty\frac{1}{y^2}dy=\frac{3}{2}\sum_{q=2}^\infty\int_q^{q+1}\frac{1}{y^2}dy
\geq\sum_{q=2}^\infty \frac{1}{q^2}$\\
Therefore, $1-\displaystyle\sum_q\frac{1}{q^2}\geq 1-\frac{3}{4}=\frac{1}{4}$, hence
equation 5.58 holds.
\subsection*{Exercise 5.17}
1) $N=a^b$, taking $\log$ of both sides\\
$L=b\log{a}$\\
If $a=1$, then $L=1$ and $b=0$.\\
If $a\geq 2$, then $\log{a}\geq 1$, hence as $b$ is a positive integer, $b\leq L$.\\
2) We want to calculate $2$ estimates to $x=\log{N}/b$, we need $O(1)$ to find $y$ 
$O(L^2)$ to calculate $x$ for a specific $b\leq L$ and $O(1)$ for calculating $2^x$ and finding the
closest 2 integers.\\
3) To calculate
\subsection*{Exercise 5.18}
$N$ is not even so step 1 is passed, using the algorithm of the exercise 5.17 
\subsection*{Exercise 5.19}
The only non composite odd number less than $15$ is $9$ which is $3^2$, hence as
$15=3*5$ it's the smallest composite number that's odd and not a perfect power. 
\subsection*{Exercise 5.20}

(Correction for the hint, $\sqrt{N/r}\rightarrow N/r$)\\
For $N=nr$, we have,\\
$\hat{f}(\ell)=\displaystyle \frac{1}{\sqrt{N}}\sum_{x=0}^{N-1}e^{-2\pi i\ell x/N}f(x)=
\displaystyle \frac{1}{\sqrt{N}}\sum_{m=0}^{n-1}\sum_{x=0}^{r-1}e^{-2\pi i\ell(mr+x)/nr}f(x)=
\displaystyle \frac{1}{\sqrt{N}}\sum_{x=0}^{r-1}\sum_{m=0}^{n-1}e^{-2\pi i\ell m/n}e^{-2\pi i\ell x/N}f(x)=
\displaystyle \frac{1}{\sqrt{N}}\sum_{x=0}^{r-1}n\delta_{\ell,zn}e^{-2\pi i\ell x/N}f(x)=
\begin{cases}
    \displaystyle \sqrt{\frac{n}{r}}\sum_{x=0}^{r-1}e^{-2\pi i\ell x/N}f(x) & \text{for }\ell=zn \text{ where } z\in\mathbb{Z}\\
    0&\text{otherwise}
\end{cases}$\\
Equation 5.63 is the fourier transform for a single period of $f(x)$.
\subsection*{Exercise 5.21}
1)$U_y\ket{\hat{f}(\ell)}=\displaystyle \frac{1}{\sqrt{r}}\sum_{x=0}^{r-1}e^{-2\pi i\ell x/N}\ket{f(x+y)}=
\displaystyle e^{2\pi i\ell y/N}\frac{1}{\sqrt{r}}\sum_{x=0}^{r-1}e^{-2\pi i\ell (x+y)/N}\ket{f(x+y)}=\\
\displaystyle e^{2\pi i\ell y/N}\frac{1}{\sqrt{r}}\sum_{x=0}^{r-1}e^{-2\pi i\ell x/N}\ket{f(x)}=
\displaystyle e^{2\pi i\ell y/N}\ket{\hat{f}(\ell)}$\\
2)$\ket{f(x_0)}=\displaystyle\frac{1}{\sqrt{r}}\sum_{\ell=0}^{r-1}e^{2\pi i \ell x_0/r}\ket{\hat{f}(\ell)}$\\
$\displaystyle \frac{1}{\sqrt{2^t}}\sum_{x=0}^{2^t-1}\ket{x}U_y\ket{f(x_0)}=
\displaystyle\frac{1}{\sqrt{2^tr}}\sum_{\ell=0}^{r-1}\sum_{x=0}^{2^t-1}e^{2\pi i \ell x_0/r}e^{2\pi i\ell y/N}\ket{x}\ket{\hat{f}(\ell)}\\
\xrightarrow{FT^\dagger}\displaystyle\frac{1}{\sqrt{r}}\sum_{\ell=0}^{r-1}e^{2\pi i\ell y/N}\ket{\tilde{\ell/r}}\ket{\hat{f}(\ell)}$\\
Which due to the equal superposition of the $\ket{\hat{f}(\ell)}$ gives the result from
phase estimation.
\subsection*{Exercise 5.22}
Using the fact that $\ket{f(x_1,x_2)}=\ket{f(0,x_2+sx_1)}$ from periodicity.\\

$\ket{\hat{f}(\ell_1,\ell_2)}=\displaystyle\frac{1}{\sqrt{r}}\sum_{x_1=0}^{r-1}
e^{-2\pi i\ell_1x_1/r}\frac{1}{\sqrt{r}}\sum_{x_2=0}^{r-1}
e^{-2\pi i\ell_2x_2/r}\ket{f(x_1,x_2)}=
\displaystyle\frac{1}{r}\sum_{x_1=0}^{r-1}\sum_{x_2=0}^{r-1}
e^{-2\pi i(\ell_1x_1+\ell_2x_2)/r}\ket{f(x_1,x_2)}=
\displaystyle\frac{1}{r}\sum_{x_1=0}^{r-1}\sum_{x_2=0}^{r-1}
e^{-2\pi i(\ell_1x_1+\ell_2x_2)/r}\ket{f(0,x_2+sx_1)}=
\displaystyle\frac{1}{r}\sum_{x_1=0}^{r-1}\sum_{j=sx_1}^{r-1+sx_1}
e^{-2\pi i(\ell_1x_1+\ell_2(j-sx_1))/r}\ket{f(0,j)}=\\
\displaystyle\frac{1}{r}\sum_{x_1=0}^{r-1}e^{-2\pi isx_1(\ell_1/s-\ell_2)/r}\sum_{j=sx_1}^{r-1+sx_1}
e^{-2\pi i\ell_2j/r}\ket{f(0,j)}=
\sum_{j=0}^{r-1}e^{-2\pi i\ell_2j/r}\ket{f(0,j)}$\\
when $\ell_1/s-\ell_2\in \mathbb{Z}$.
\subsection*{Exercise 5.23}
Should be a $+$ in the exponent.\\
Using $\ell_1=\ell_2s+nrs$\\
$\displaystyle\frac{1}{r}\sum_{\ell_1=0}^{r-1}\sum_{\ell_2=0}^{r-1}
e^{2\pi i(\ell_1x_1+\ell_2x_2)/r}\ket{\hat{f}(\ell_1,\ell_2)}=
\displaystyle\frac{1}{r}\sum_{\ell_1=0}^{r-1}\sum_{\ell_2=0}^{r-1}
e^{2\pi i(\ell_1x_1+\ell_2x_2)/r}\sum_{j=0}^{r-1}e^{-2\pi i\ell_2j/r}\ket{f(0,j)}=\\
\displaystyle\frac{1}{r}\sum_{\ell_2=0}^{r-1}\sum_{j=0}^{r-1}
e^{2\pi i((\ell_2s+nrs)x_1+\ell_2(x_2-j))/r}\ket{f(0,j)}=
\displaystyle\frac{1}{r}\sum_{\ell_2=0}^{r-1}\sum_{j=0}^{r-1}
e^{2\pi i\ell_2(sx_1+x_2-j)/r}\ket{f(0,j)}=\\
\displaystyle\sum_{j=0}^{r-1}
\delta_{x_2+sx_1,j}\ket{f(0,j)}=\ket{f(0,x_2+sx_1)}=\ket{f(x_1,x_2)}$
\subsection*{Exercise 5.24}
Let $\varphi_1=\widetilde{sl_2/r}$ and $\varphi_2=\widetilde{l_2/r}$. Both are $t$ bits long, hence
$\exists$ $sl_2/r$ and $l_2/r$ which are the convergents of the continued fractions
$\varphi_1$ and $\varphi_2$ respectively, if\\
$\displaystyle\left|\frac{sl_2}{r}-\varphi_1\right|\leq\frac{1}{2r^2}$\\
$\displaystyle\left|\frac{l_2}{r}-\varphi_2\right|\leq\frac{1}{2r^2}$\\
Therefore, with the algorithm of continued fractions we can find $sl_2/r$ and $l_2/r$,
and hence $s$ by dividing one by the other.
\subsection*{Exercise 5.25}

\subsection*{Exercise 5.26}
\subsection*{Exercise 5.27}
\subsection*{Exercise 5.28}
\subsection*{Exercise 5.29}
\end{document}