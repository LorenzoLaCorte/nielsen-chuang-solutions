\documentclass[a4paper,12pt]{article}
\usepackage{amsmath}
\usepackage[margin=0.9in]{geometry}
\usepackage{braket}
\begin{document}
\subsection*{Exercise 10.1}
Let $\ket{\psi}=a\ket{0}+b\ket{1}$ and the initial state be $\ket{\psi_0}=a\ket{000}+b\ket{100}$.\\
Applying a CNOT to the first two qubits we get,\\
$\ket{\psi_1}=a\ket{000}+b\ket{110}$\\
Applying a CNOT to the first and last qubits we get,\\
$\ket{\psi_2}=a\ket{000}+b\ket{111}$
\subsection*{Exercise 10.2}
$P_\pm=\frac{1}{2}(\ket{0}\pm\ket{1})(\bra{0}\pm\bra{1})=
\frac{1}{2}(\ket{0}\bra{0}+\ket{1}\bra{1}\pm\ket{1}\bra{0}\pm\ket{0}\bra{1})
=\frac{1}{2}(I\pm X)$\\
Therefore,\\
$\mathcal{E}(\rho)=(1-2p)\rho+2pP_+\rho P_++2pP_-\rho P_-=
(1-2p)\rho+\frac{1}{2}p(I+X)\rho(I+X)+\frac{1}{2}p(I-X)\rho(I-X)=
(1-2p)\rho+p\rho+pX\rho X=(1-p)\rho+pX\rho X$
\subsection*{Exercise 10.3}
$Z_2Z_3Z_1Z_2=[I\otimes (\ket{00}\bra{00}+\ket{11}\bra{11})-I\otimes (\ket{01}\bra{01}+\ket{10}\bra{10})]
[(\ket{00}\bra{00}+\ket{11}\bra{11})\otimes I-(\ket{01}\bra{01}+\ket{10}\bra{10})\otimes I]=
\underbrace{\ket{000}\bra{000}+\ket{111}\bra{111}}_{P_0}
-(\underbrace{\ket{100}\bra{100}+\ket{011}\bra{011}}_{P_1})\\
+\underbrace{\ket{010}\bra{010}+\ket{101}\bra{101}}_{P_2}
-(\underbrace{\ket{001}\bra{001}+\ket{110}\bra{110}}_{P_3})$
\subsection*{Exercise 10.4}
1)$\ket{000}\bra{000}$, $\ket{111}\bra{111}$: no bit flip\\
$\ket{100}\bra{100}$, $\ket{011}\bra{011}$: first bit flipped\\
$\ket{010}\bra{010}$, $\ket{101}\bra{101}$: second bit flipped\\
$\ket{001}\bra{001}$, $\ket{110}\bra{110}$: third bit flipped\\
2) If our state is $\ket{\psi}=a\ket{000}+b\ket{111}$, then the measurement will
collapse the state into $\ket{000}$ or $\ket{111}$ with probabilities $|a|^2$ or $|b|^2$,
respectively. Hence, only the computational basis states $\ket{000}$ and $\ket{111}$ can 
be corrected.\\
3)Assuming the initial state is $\ket{000}$ the probability that one or fewer bit flips occur is $(1-p)^3+p(1-p)^2$, hence
$F\geq\sqrt{(1-p)^3+p(1-p)^2}$.
\subsection*{Exercise 10.5}
Assuming no more than one error has occurred, $X_1X_2X_3X_4X_5X_6$ will be $1$ if no phase flip occurred
and $-1$ and if one occurred on the first or second block. Identically for $X_4X_5X_6X_7X_8X_9$.
Hence, if both give $-1$ the error is on the second block, otherwise it's on the first block if
$X_1X_2X_3X_4X_5X_6$ gives $-1$ and on the third block if $X_4X_5X_6X_7X_8X_9$ gives $-1$.
If both give $1$ then no error has occurred.
\subsection*{Exercise 10.6}
The eigenvalues of $Z$ are $\pm 1$, hence\\
$Z_1Z_2Z_3(\ket{000}-\ket{111})=\ket{000}-(-1)^3\ket{111}=\ket{000}+\ket{111}$
\subsection*{Exercise 10.7}
Need to prove that $PE_i^\dagger E_jP=\alpha_{ij}P$. $I$ and $X$ are Hermitian, hence suffices
to show for $IX_1$,$II$,$X_1X_1$ and $X_1X_2$.\\
$P\sqrt{(1-p)^3}I\sqrt{p(1-p)^2}X_1P=(1-p)^2\sqrt{p(1-p)}
(\ket{000}\bra{000}+\ket{111}\bra{111})X_1(\ket{000}\bra{000}+\ket{111}\bra{111})=
(1-p)^2\sqrt{p(1-p)}
(\ket{000}\bra{000}+\ket{111}\bra{111})(\ket{100}\bra{000}+\ket{011}\bra{111})=0$\\
$P\sqrt{(1-p)^3}I\sqrt{(1-p)^3}IP=(1-p)^3PP=(1-p)^3P$\\
$P\sqrt{p(1-p)^2}X_1\sqrt{p(1-p)^2}X_1P=p(1-p)^2PIP=p(1-p)^2P$\\
$P\sqrt{p(1-p)^2}X_1\sqrt{p(1-p)^2}X_2=p(1-p)^2
(\ket{000}\bra{000}+\ket{111}\bra{111})(\ket{110}\bra{000}+\ket{001}\bra{111})=0$\\
Hence, the quantum error-correction conditions are satisfied.
\subsection*{Exercise 10.8}
$P=\ket{+++}\bra{+++}+\ket{---}\bra{---}$, hence like in the previous exercise.\\
$PE_i^\dagger E_jP=0$, $i\neq j$\\
$PE_i^\dagger E_jP=P$, $i=j$\\
Hence, the quantum error-correction conditions are satisfied.
\subsection*{Exercise 10.9}
$PIIP=P$\\
$PIP_1P=(\ket{+++}\bra{+++}+\ket{---}\bra{---})(\ket{0}\bra{0}\otimes I\otimes I)
(\ket{+++}\bra{+++}+\ket{---}\bra{---})=
(\ket{+++}\bra{+++}+\ket{---}\bra{---})\frac{1}{\sqrt{2}}(\ket{0++}\bra{+++}+\ket{0--}\bra{---})=
\frac{1}{2}(\ket{+++}\bra{+++}+\ket{---}\bra{---})=\frac{1}{2}P$\\
Identically,\\
$PIQ_1P=\frac{1}{2}P$\\
$PP_1Q_1=0$\\
$PP_1P_1P=PP_1P=\frac{1}{2}P$\\
$PQ_1Q_1P=PQ_1P=\frac{1}{2}P$\\
$PP_1P_2P=(\ket{+++}\bra{+++}+\ket{---}\bra{---})(\ket{0}\bra{0}\otimes \ket{0}\bra{0}\otimes I)
(\ket{+++}\bra{+++}+\ket{---}\bra{---})=
(\ket{+++}\bra{+++}+\ket{---}\bra{---})\frac{1}{2}(\ket{00+}\bra{+++}+\ket{00-}\bra{---})=
\frac{1}{4}(\ket{+++}\bra{+++}+\ket{---}\bra{---})=\frac{1}{4}P$\\
$PP_1Q_2P=(\ket{+++}\bra{+++}+\ket{---}\bra{---})(\ket{0}\bra{0}\otimes \ket{1}\bra{1}\otimes I)
(\ket{+++}\bra{+++}+\ket{---}\bra{---})=
(\ket{+++}\bra{+++}+\ket{---}\bra{---})\frac{1}{2}(\ket{01+}\bra{+++}-\ket{01-}\bra{---})
=\frac{1}{4}(\ket{+++}\bra{+++}+\ket{---}\bra{---})=\frac{1}{4}P$\\
Hence, the quantum error-correction conditions are satisfied.
\subsection*{Exercise 10.10}
$P=\ket{0_L}\bra{0_L}+\ket{1_L}\bra{1_L}$\\
Due to phase and bit flips,\\
$PIX_iP=PIY_iP=PIZ_iP=0$\\
$PIIP=PX_iX_iP=PY_iY_iP=PZ_iZ_iP=P$\\
The $X_i$ and $Y_i$ change the individual qubits, hence if $i\neq j$ $PX_iY_jP=0$, e.g.
for $PX_1Y_2P$ looking at the first triplet, we have\\
$(\bra{000}+\bra{111})i(\ket{110}-\ket{001})=0$\\
$X_iY_i=iZ_i$, hence $PX_iY_iP=0$\\
For $Z_iZ_j$ if $i$ and $j$ belong to different triplets then we have a phase flip on $2$
separate triplets, hence $PZ_iZ_jP=0$. \\
However, if $i$ and $j$ are in the same triplet, then
we apply $2$ phase shifts to the triplet which is equivalent to no change, hence
$PZ_iZ_jP=P$.\\
For $X_iZ_j$ and $Y_iZ_j$ we perform a bit and phase flip, hence for all $i$ and $j$
$PX_iZ_jP=PY_iZ_jP=0$.

\subsection*{Exercise 10.11}
$\mathcal{E}(\rho)=\frac{I}{2}$\\
Consider the operation elements found for the general depolarizing channel in Exercise
8.19 $\{\sqrt{\frac{p}{d}}\ket{i}\bra{j}\}$. Taking $p=1$ and $d=2$, we get 
$\{\frac{1}{2}\ket{0}\bra{0},\frac{1}{2}\ket{1}\bra{1},\frac{1}{2}\ket{0}\bra{1},\frac{1}{2}\ket{1}\bra{0}\}$.
\subsection*{Exercise 10.12}
$F(\ket{0},\mathcal{E}(\ket{0}\bra{0}))=\sqrt{\bra{0}\mathcal{E}(\ket{0}\bra{0})\ket{0}}\\
=\sqrt{\bra{0}((1-p)\ket{0}\bra{0}+\frac{p}{3}(X\ket{0}\bra{0}X+Y\ket{0}\bra{0}+Z\ket{0}\bra{0}Z))\ket{0}}=
\sqrt{1-p+\frac{p}{3}}=\sqrt{1-\frac{2p}{3}}$
As the depolarizing channel is symmetric, for any pure state $\ket{\psi}$,\\ 
$F(\ket{\psi},\mathcal{E}(\ket{\psi}\bra{\psi}))=\sqrt{1-\frac{2p}{3}}$. \\As fidelity is
jointly concave, for any $\rho$ and some $\ket{\psi}$ we have,\\
$F(\rho, \mathcal{E}(\rho))\geq F(\ket{\psi},\mathcal{E}(\ket{\psi}\bra{\psi}))=\sqrt{1-\frac{2p}{3}}$
\subsection*{Exercise 10.13}
Let $\ket{\psi}=a\ket{0}+b\ket{1}$\\
$F(\ket{\psi},\mathcal{E}(\ket{\psi}\bra{\psi}))=\sqrt{\bra{\psi}\mathcal{E}(\ket{\psi}\bra{\psi})\ket{\psi}}
\\
\sqrt{|\bra{\psi}E_0\ket{\psi}|^2+|\bra{\psi}E_1\ket{\psi}|^2}=
\sqrt{||a|^2+|b|^2\sqrt{1-\gamma}|^2+|a|b|^2\sqrt{\gamma}|^2}$\\
Minimum will occur when $a=0$ and $b=1$, hence\\
$F_{min}(\ket{\psi},\mathcal{E}(\ket{\psi}\bra{\psi}))=F(\ket{1},\mathcal{E}(\ket{1}\bra{1}))=\sqrt{1-\gamma}$
\subsection*{Exercise 10.14}
$G=
rk\underbrace{\left\{
\begin{bmatrix}
    1 & 0 & \ldots & 0\\
    \scriptstyle{r}\vdots & \vdots & \vdots & \vdots\\
    1 & 0 & \ldots & 0\\
    0 & 1 & \ldots & 0\\
    \vdots & \vdots & \vdots & \vdots\\
    0 & 1 & \ldots & 0\\
    \vdots & \vdots & \vdots & \vdots\\
    0 & 0 & \ldots & 1\\
    \vdots & \vdots & \vdots & \vdots\\
    0 & 0 & \ldots & 1
    
\end{bmatrix}\right.}_{k}$
\newpage
\subsection*{Exercise 10.15}
Let $c_1$ and $c_2$ be columns of $G$. Then\\
$G=[c_1|c_2|G^\prime]\\
G^{\prime\prime}=[c_1|c_1+c_2|G^\prime]$\\
Let $x=(x_1,x_2,\ldots, x_n)$.\\
$Gx=c_1x_1+c_2x_2+\ldots$\\
$G^{\prime\prime}x=c_1x_1+(c_1+c_2)x_2+\ldots$\\
$G^{\prime\prime}x-Gx=c_1x_2\in C$\\
Therefore, as $C$ is linear with $G$ as generator, $G^{\prime\prime}$ is a generator for
$C$ as well, as the difference of the two codes is still in $C$.  
\subsection*{Exercise 10.16}
Let $r_1$ and $r_2$ be rows of $H$. Then\\
$H=\left[\begin{array}{c}
    r_1 \\ \hline
    r_2 \\ \hline
    H^\prime
    \end{array}\right]$\\
$H^{\prime\prime}=\left[\begin{array}{c}
    r_1 \\ \hline
    r_1+r_2 \\ \hline
    H^\prime
    \end{array}\right]$\\
Let $x=(x_1,x_2,\ldots, x_n)$.\\
$Hx=\begin{bmatrix}
    r_1x\\
    r_2x\\
    \vdots
\end{bmatrix}=0$\\
Therefore, $r_1x=r_2x=0$. Hence,\\
$H^{\prime\prime}x=\begin{bmatrix}
    r_1x\\
    r_1x+r_2x\\
    \vdots
\end{bmatrix}=0$\\
Hence, $H^{\prime\prime}$ is a parity check matrix for the same code.
\subsection*{Exercise 10.17}
$y_1=(1,1,1,0,0,0)$, $y_2=(0,0,0,1,1,1)$, hence we can take $y_3$ to $y_6$ as,\\
$y_3=(1,1,0,0,0,0)$\\
$y_4=(1,0,1,0,0,0)$\\
$y_5=(0,0,0,0,1,1)$\\
$y_6=(0,0,0,1,0,1)$\\
Therefore,\\
$H=\begin{bmatrix}
    1&1&0&0&0&0\\
    1&0&1&0&0&0\\
    0&0&0&0&1&1\\
    0&0&0&1&0&1\\
\end{bmatrix}$
\newpage
\subsection*{Exercise 10.18}
Let $x$ be an arbitrary message to be encoded. Then,\\
$y=Gx\in C$\\
Hence,
$HGx=Hy=0$ for $\forall x$\\
Hence,
$HG=0$
\subsection*{Exercise 10.19}
Using that $HG=0$ we have,\\
$HG=\begin{bmatrix}
    a_{11}& a_{12} &\ldots&a_{1k}&1& \ldots &0\\
    \vdots &\vdots &\vdots &\vdots & &\ddots &\\
    a_{(n-k)1}& a_{(n-k)2} &\ldots&a_{(n-k)k}&0 &\ldots &1
\end{bmatrix}
\begin{bmatrix}
    b_{11}&b_{12}&\ldots&b_{1k}\\
    \vdots &\vdots &\vdots &\vdots\\
    b_{n1}&b_{n2}&\ldots&b_{nk}
\end{bmatrix}=0$\\
Hence,\\
$\displaystyle\sum_{i\leq k}a_{1i}b_{i1}+b_{(k+1)1}=0
\ldots
\displaystyle\sum_{i\leq k}a_{(n-k)i}b_{i1}+b_{n1}=0\\
\vdots\\
\displaystyle\sum_{i\leq k}a_{1i}b_{ik}+b_{(k+1)k}=0
\ldots
\displaystyle\sum_{i\leq k}a_{(n-k)i}b_{ik}+b_{nk}=0$\\
We see that for example, taking for $2\leq i \leq k$ $b_{i1}=0$ , $b_{11}=1$ and $b_{(k+1)1}=-a_{11}$
gives a solution.\\
Therefore for $i,j\leq k$ $b_{ij}=\delta_{ij}$ and for $i,j>k$ $b_{ij}=-a_{(i-k)j}$, i.e.\\
$G=\left[\begin{array}{c}
    I_k\\ \hline
    -A
    \end{array}\right]$
\subsection*{Exercise 10.20}
Let $x$ be a codeword such that wt$(x)\leq d-1$. Let $H={c_1|c_2\ldots c_n}$ for code $C$. Consider $Hx$,\\
$Hx=\displaystyle \sum_ic_ix_i$ for $d-1$ columns. Therefore, as any $d-1$ columns are linearly
independent, this sum cannot equal $0$. Hence, $d(C)\geq d$. However, as any $d$ columns
are linearly dependant there exists a codeword $y$ with wt$(y)=d$ such that $Hy=0$. Therefore,
$d(C)=d$.
\subsection*{Exercise 10.21}
The parity check matrix is a $n-k$ by $n$ matrix, hence the maximum number of linearly independent
columns is $n-k$. Therefore, from Exercise 10.20 $n-k\geq d-1$.
\subsection*{Exercise 10.22}
The Hamming parity check matrix is constructed from columns which are all the possible
$n-k$ bit strings, of which there are $2^r-1$ of excluding the $0$ string. Hence, any
two columns will be linearly independent as all are different, however there always will be $3$
linearly dependant columns, e.g. $(1,0,0,\ldots)$, $(0,1,0,\ldots)$ and $(1,1,0,\ldots)$.
Therefore, as per exercise 10.20 the code will have distance $3$. 
\subsection*{Exercise 10.23}

\subsection*{Exercise 10.24}
\subsection*{Exercise 10.25}
\subsection*{Exercise 10.26}

\end{document}